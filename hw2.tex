\documentclass[12pt]{article}
 \usepackage[margin=1in]{geometry} 
\usepackage{amsmath,amsthm,amssymb,amsfonts}
\usepackage{mathtools, eucal}
 

 
\begin{document}
 
%\renewcommand{\qedsymbol}{\filledbox}
%Good resources for looking up how to do stuff:
%Binary operators: http://www.access2science.com/latex/Binary.html
%General help: http://en.wikibooks.org/wiki/LaTeX/Mathematics
%Or just google stuff
 
\title{Homework 2 Solutions}
\author{Zheming Gao}
\maketitle

\section*{Problem 1.1}

Solutions: 
\begin{enumerate}
\item 

Let $x_2 = x_2^+ - x_2^-$ and $x_2 ^+, x_2^- \geqslant 0$.
And add slack variables $\xi_1, \xi_2$ on the first and the second constraint respectively.

\begin{equation*}
\begin{aligned}
\text{Minimize} \quad & 4x_1 + \sqrt{2}x_2^+ - \sqrt{2}x_2^- - 0.35 x_3 \\
\text{subject\  to} \quad & -0.001x_1 + 200 x_2^+ - 200 x_2^- - \xi_1 = 7\sqrt{261} \\
& 7.07 x_2^+ - 7.07 x_2^- - 2.62 x_3 + \xi_2 = -4 \\
& x_1, x_2^+, x_2^-, x_3, \xi_1, \xi_2 \geqslant 0
\end{aligned}
\end{equation*}

\item

Let $a_1 = x_1 - 20$, $a_3 = x_3 + 15$, then $a_1, a_3 \geqslant 0$, and $x_1 = a_1 + 20, x_3 = a_3 - 15$.

Add a slack variable $\xi_1$ on the second constraint. The standard form is

\begin{equation*}
\begin{aligned}
\text{Minimize} \quad & 3.1a_1 - 2\sqrt{2}x_2 + a_3 + 47 \\
\text{subject\  to} \quad & 100a_1 - 20 x_2 = -1993 \\
& -11a_1 - 7\pi x_2 - 2a_3 + \xi_1 = 590\\
& a_1, x_2, a_3, \xi_1 \geqslant 0
\end{aligned}
\end{equation*}

\item

Since $x_3 \leqslant 10$, $10 - x_3 \geqslant 0$. Let $a_3 = 10 - x_3$, then $a_3 \geqslant 0$ and $x_3 = 10 - a_3$.
Let $x_1 = x_1^+ - x_1^-$, where $x_1^+, x_1^- \geqslant 0$. Add slack variables on each constraint and the standard form is the following,

\begin{equation*}
\begin{aligned}
\text{Minimize} \quad & -x_1^+ + x_1^- - 3x_2 - 2a_3 + 20 \\
\text{subject\  to} \quad & 3x_1^+ - 3x_1^- - 5x_2 - \xi_1 = -2 \\
& 3x_1^+ - 3x_1^- - 5x_2 + \xi_2 = 15\\
& -5x_1^+ + 5x_1^- + 20x_2 - \xi_3 = 11 \\
& -5x_1^+ + 5x_1^- + 20x_2 + \xi_4 = 40 \\
& x_1^+, x_1^-, x_2, a_3, \xi_i (i = 1,\dots, 4)  \geqslant 0
\end{aligned}
\end{equation*}

\end{enumerate}

\section*{Problem 1.2}
\begin{enumerate}
\item[a)]

Let $x_1 = x_1^+ - x_1^-$, where $x_1^+, x_1^- \geqslant 0$. The standard form is the following,

\begin{equation*}
\begin{aligned}
\text{Minimize} \quad & 2x_1^+ - 2x_1^- + 6x_2 + 8x_3 \\
\text{subject\  to} \quad & x_1^+ - x_1^- + 2x_2 + x_3 = 5 \\
& 4x_1^+ - 4x_1^-  + 2x_3 = 12\\
& x_1^+, x_1^-, x_2, x_3 \geqslant 0
\end{aligned}
\end{equation*}

\item[b)]
From the first constraint, solve $x_1$ as $x_1 = 5 - 2x_2 - x_3$. Plug it into the objective function and also the second constraint, then reform the LP problem as the following,

\begin{equation*}
\begin{aligned}
\text{Minimize} \quad & 2x_2 + 6x_3 +10\\
\text{subject\  to} \quad & -2x_2 - 2x_3 = -8 \\
& x_2, x_3 \geqslant 0
\end{aligned}
\end{equation*}

\item[c)]
It is already in the standard form.

\item[d)]

Use graphic method to solve the problem. Optimal value is $z* = 10$ and the optimal solution is $x* = (x_2^*, x_3^*) = (0, 0)$.
\end{enumerate}


\section*{Problem 1.3}

\begin{enumerate}
\item[a)]

No. Because there is a nonlinear term $x_1^2$ in the objective function and the first constraint.

\item[b)]

Yes. Use the first constraint, solve $x_1^2$ and get $x_1^2 = x_2$. Plug it into the objective function and get 

\begin{equation*}
\begin{aligned}
\text{Minimize} \quad & 2x_2 + 4x_3 \\
\text{subject\  to} \quad & 2x_2 + 4x_3 \geqslant 4 \\
& x_1, x_3 \geqslant 0, x_2 \geqslant 2
\end{aligned}
\end{equation*}


\item[c)]

Use the similar technique in problem 1.1 and 1.2. Let $a_2 = x_2 - 2$. Then $a_2 \geqslant 0$, and $x_2 = a_2 + 2$. Add a slack variable on the constraint. 

\begin{equation*}
\begin{aligned}
\text{Minimize} \quad & 2a_2 + 4x_3 + 4 \\
\text{subject\  to} \quad & 2a_2 + 4x_3 - \xi = 0 \\
& x_1, a_2, x_3, \xi \geqslant 0
\end{aligned}
\end{equation*}

\item[d)]

Yes. To solve the LP problem, use graphic method. 










To solve the original problem, we can graph the feasible region in 3-D and use graph method to solve it.


\end{enumerate}

\section*{Problem 1.4}
\begin{enumerate}
\item[a)]

No. Because it has absolute-value functions in the objective function.

\item[b)]
Let $x_i = x_i^+ - x_i^-$, $i = 1, 2, 3$, where $x_i^+, x_i^- \geqslant 0$. Then $|x_i| = x_i^+ + x_i^-$. Add one slack variable $\xi_1$ on the first constraint and the problem is reformed as

\begin{equation*}
\begin{aligned}
\text{Minimize} \quad & x_1^+ + x_1^- + 2x_2^+ + 2x_2^- - x_3^+ - x_3^- \\
\text{subject\  to} \quad & x_1^+ - x_1^- + x_2^+ - x_2^- - x_3^+ + x_3^- + \xi_1 = 10 \\
& x_1^+ - x_1^- - 3x_2^+ + 3x_2^- + 2x_3^+ - 2x_3^- = 12 \\
& x_i^+, x_i^- \geqslant 0, i = 1, 2, 3. \\
& \xi_1 \geqslant 0
\end{aligned}
\end{equation*}

\item[c)]

Use the similar technique as in (b), let $a_1 = x_1 - 5$ and $a_2 = x_2 + 4$. Then, $|x_1 - 5| = a_1^+ + a_1^-$ and $ |x_2 + 4| = a_2^+ + a_2^-$. 

Also, it is clear to see that

$$
x_1 = a_1 + 5 = a_1^+ - a_1^- + 5, \quad x_2 = a_2 - 4 = a_2^+ - a_2^- - 4.
$$

Plug above into the problem and get the following standard form.

\begin{equation*}
\begin{aligned}
\text{Minimize} \quad & a_1^+ + a_1^- + a_2^+ + a_2^- \\
\text{subject\  to} \quad & a_1^+ - a_1^- + a_2^+ - a_2^- + \xi_1 = 10 \\
& a_1^+ - a_1^- - 3a_2^+ + 3a_2^- - \xi_2 = -15 \\
& a_i^+, a_i^-, \xi_i \geqslant 0, i = 1, 2. 
\end{aligned}
\end{equation*}
\end{enumerate}


\section*{Problem 1.5}



\end{document}