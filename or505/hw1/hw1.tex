\documentclass[12pt]{article}
 \usepackage[margin=1in]{geometry} 
\usepackage{amsmath,amsthm,amssymb,amsfonts}
\usepackage{mathtools, eucal}
 

 
\begin{document}
 
%\renewcommand{\qedsymbol}{\filledbox}
%Good resources for looking up how to do stuff:
%Binary operators: http://www.access2science.com/latex/Binary.html
%General help: http://en.wikibooks.org/wiki/LaTeX/Mathematics
%Or just google stuff
 
\title{Homework 1 Solutions}
\author{Zheming Gao}
\maketitle

\section{Problem 1}

Solutions: 
\begin{enumerate}
\item 
\begin{equation*}
det
\begin{pmatrix}
a^2 & ab \\
ab & b^2
\end{pmatrix}
= a^2b^2 - abab = 0.
\end{equation*}

\item
\begin{equation*}
det
\begin{pmatrix}
\cos\alpha & -\sin\alpha \\
\sin\alpha & \cos\alpha
\end{pmatrix}
= \cos^2\alpha + \sin^2\alpha = 1.
\end{equation*}

\item
\begin{equation*}
det
\begin{pmatrix}
1 & x & x \\
x & 2 & x \\
x & x & 3
\end{pmatrix}
= 1\times 2 \times 3 + x^3 + x^3 - 2x^2 - x^2 - 3x^2 = 
2x^3 - 6x^2 + 6.
\end{equation*}

\item
Similar to the problem above, 
\begin{equation*}
\begin{aligned}
det
\begin{pmatrix}
1 & 1 & 1 \\
a & b & c \\
a^2 & b^2 & c^2
\end{pmatrix}
& = bc^2 + a^2c + ab^2 - a^2b - ac^2 - b^2c \\
& = ab(b-a) + ac(a-c) + bc(c-b).
\end{aligned}
\end{equation*}

\item
Change the order of columns. And apply the operations on matrix blocks:
\begin{equation*}
\begin{aligned}
det
\begin{pmatrix}
1 & 0 & 2 & a\\
2 & 0 & b & 0 \\
3 & c & 4 & 5 \\
d & 0 & 0 & 0
\end{pmatrix}
& = 
- det
\begin{pmatrix}
0 & 2 & a & 1\\
 0 & b & 0 & 2 \\
c & 4 & 5 & 3\\
0 & 0 & 0 & d
\end{pmatrix}
 \\
& = 
-det 
\begin{pmatrix}
0 & 2 & a\\
0 & b & 0 \\
c & 4 & 5
\end{pmatrix}
\cdot d \\
& = abcd.
\end{aligned}
\end{equation*}

\item
Same idea as the problem above. Note that it won't change the determinant of a matrix if we add one row multiplied by a constant number on another row. 

suppose $a\neq 0$. Then, add $\frac{1}{a}$row1 on row2. The determinant doesn't change.
\begin{equation*}
det
\begin{pmatrix}
a & 1 & 0 & 0\\
-1 & b & 1 & 0 \\
0 & -1 & c & 1 \\
0 & 0 & -1 & d
\end{pmatrix}
= 
det
\begin{pmatrix}
a & 1 & 0 & 0\\
0 & b + \frac{1}{a} & 1 & 0 \\
0 & -1 & c & 1 \\
0 & 0 & -1 & d
\end{pmatrix}
\end{equation*}

Then change orders of rows, and change orders of columns. 
\begin{equation*}
\begin{aligned}
det
\begin{pmatrix}
a & 1 & 0 & 0\\
0 & b + \frac{1}{a} & 1 & 0 \\
0 & -1 & c & 1 \\
0 & 0 & -1 & d
\end{pmatrix}
& = 
- det
\begin{pmatrix}
0 & b + \frac{1}{a} & 1 & 0 \\
0 & -1 & c & 1 \\
0 & 0 & -1 & d \\
a & 1 & 0 & 0
\end{pmatrix} \\
& = 
det
\begin{pmatrix}
b + \frac{1}{a} & 1 & 0 & 0 \\
-1 & c & 1 & 0 \\
0 & -1 & d & 0\\
1 & 0 & 0 & a
\end{pmatrix} \\
& = 
det 
\begin{pmatrix}
b + \frac{1}{a} & 1 & 0\\
-1 & c & 1 \\
0 & -1 & d
\end{pmatrix} 
\cdot a \\
& = 1 + ab + ad + cd + abcd.
\end{aligned}
\end{equation*}

\end{enumerate}


\section{Problem 2}

Solution:
All linear systems can be formed as $Ax = b$. 
\begin{enumerate}
\item
In this problem, $x \in \mathbb R^4$, $A$ is a 4 by 4 matrix and $b$ is a 4 by 1 vector.

\begin{equation*}
[A|b] = \begin{bmatrix}
2 & -1/2 & -1/2 & 0 & 0 \\
-1/2 & 2 & 0 & -1/2 & 3 \\
-1/2 & 0 & 2 & -1/2 & 3 \\
0 & -1/2 & -1/2 & 2 & 0 
\end{bmatrix} \xrightarrow {Gauss \ \ elimination}
\begin{bmatrix}
1 & -1/4 & -1/4 & 0 & 0 \\
0 & 1 & -1/15 & -4/15 & 8/5 \\
0 & 0 & 1 & -2/7 & 12/7 \\
0 & 0 & 0 & 1 & 1 
\end{bmatrix}
\end{equation*}

Thus, the solution to the linear system is $x = (1, 2, 2, 1)^T$.

\item
\begin{equation*}
[A|b] = \begin{bmatrix}
2 & 3 & 5 & 1 & 3 \\
3 & 4 & 2 & 3 & -2 \\
1 & 2 & 28 & -1 & 8 \\
7 & 9 & 1 & 8 & 0 
\end{bmatrix} \xrightarrow {Gauss \ \ elimination}
\begin{bmatrix}
1 & 3/2 & 5/2 & 1/2 & 3/2 \\
0 & 1 & 11 & -3 & 13 \\
0 & 0 & 0 & 0 & 0 \\
0 & 0 & 0 & 0 & 1 
\end{bmatrix}
\end{equation*}

We see $[A|b]$ is not in full rank so there is no solution to the linear system.
\end{enumerate}

\section{Problem 3}
$A = P\Lambda Q$, $A^2 = P\Lambda QP\Lambda Q$. Since $QP = I$, $A^2 = P \lambda^2 Q$. Hence, $A^k = P\Lambda ^k Q$. 

It is enough to calculate what $\Lambda^k$. Since $\Lambda$ is a diagonal matrix, 
\begin{itemize}
\item k is a odd number, 

$\Lambda ^k = \Lambda$. Then 
\begin{equation*}
A^k = P\Lambda Q = \begin{pmatrix}
7 & -12 \\
6 & -7
\end{pmatrix}.
\end{equation*}
\item k is an even number, 

$\Lambda ^k = I$. Then 
\begin{equation*}
A^k = PIQ = PQ = I.
\end{equation*}

\end{itemize}

\section{Problem 4}

The linear system can be formed as $Ax = b$, where
\begin{equation*}
A = \begin{bmatrix}
1 & 1 & 1\\
0 & 2 & 2\\
1 & -1 & 0
\end{bmatrix}, \ \ 
b = \begin{bmatrix}
1 \\
1 \\
2
\end{bmatrix}
\end{equation*}

It is clear that $A$ is invertible.(You may check that A is of full rank)
To find the inverse of $A$, i.e. $A^{-1}$, there is one method by Gauss elimination

\begin{equation*}
[A|I] = \begin{bmatrix}
1 & 1 & 1 & 1 & 0 & 0 \\
0 & 2 & 2 & 0 & 1 & 0 \\
1 & -1 & 0 & 0 & 0 & 1
\end{bmatrix} \xrightarrow {Gauss \ \ elimination}
\begin{bmatrix}
1 & 0 & 0 & 1 & -1/2 & 0 \\
0 & 1 & 0 & 1 & -1/2 & -1 \\
0 & 0 & 1 & -1 & 1 & 1 
\end{bmatrix} = [I|A^{-1}]
\end{equation*}

The solution is $x = A^{-1}b = (1/2, -3/2, 2)^T$.


\section{Problem 5}
\begin{enumerate}
\item False
Counter example:

Let 
$$
A = \begin{pmatrix}
1 & 0 \\
0 & 0
\end{pmatrix} \quad
B = \begin{pmatrix}
0 & 0 \\
0 & 1
\end{pmatrix} 
$$

Neither $A$ nor $B$ is invertible, but $A + B = I$, which is invertible.

\item True

You may use different method to prove this claim. We will give two methods here. The second one is a more popular one. For those students who are also taking MA 523/723 courses, you may want to have a look at the first method.

\begin{enumerate}
\item Method 1

\begin{proof}
We use the notation $\mathcal R (A)$ on be half of "Range of matrix $A$". And there is a fact that $\mathcal R (AB) \subset \mathcal R (A)$. Hence, the dimensions of $\mathcal R (A)$ and $\mathcal R(B)$ are related in the following inequality chain:

$$
\dim(\mathcal R(AB)) \leq \dim(\mathcal R(A)) \leq n.
$$

Since $AB$ is invertible, $\dim(\mathcal R(AB)) = n$. Hence, $\dim(\mathcal R(AB)) = \dim(\mathcal R(A)) = n$, which implies two facts that
\begin{enumerate}
\item\label{fact1} $\mathcal R(A)$ is of full rank.

\item\label{fact2} $\mathcal R(A) = \mathcal R(AB)$.
\end{enumerate}

From (\ref{fact1}) we know $A$ is invertible and from (\ref{fact2}) we know that $B$ is of full rank, which also implies that $B$ is invertible.
\end{proof}

\item Method 2

\begin{proof}

Since $AB$ is invertible, we know $det(AB) \neq 0$. With $det(AB) = det(A)\cdot det(B)$, it implies $det(A)\cdot det(B) \neq 0$, which means both $det(A) \neq 0$ and $det(B) \neq 0$. This proves that both $A$ and $B$ are invertible.

\end{proof}



\end{enumerate}


\item False
Counter example:
We might let
$$
A = \begin{pmatrix}
1 & 0 \\
0 & 1
\end{pmatrix}, \quad
B = \begin{pmatrix}
1 & 0 \\
0 & 0
\end{pmatrix}
$$

Hence, $$AB = \begin{pmatrix}
1 & 0 \\
0 & 0
\end{pmatrix}
$$

Obviously, $AB$ is not invertible, but $A$ is invertible.



\item True.

\begin{proof}
Since $A$ is invertible, we know that the kernel of $A$, denoted as $\mathcal N(A) = \{0\}$. i.e., $Ax = 0$ implies $x = 0$. 

Let $kAx = 0$. Change the order of factors on left side and we obtain $A\cdot kx = 0$, which implies that $kx = 0$. With $k\neq 0$, we know that $x = 0$. In all, we conclude that $\mathcal N(kA) = \{0\}$, which is an equivalent to the fact that $kA$ is invertible. And this proves the claim.

\end{proof}
\end{enumerate}


\section{Problem 6}
Solution: (b)

\begin{proof}
Since $A = B^T$, we know $A^{-1} = B^{-T}$. Then, 

$$
\begin{aligned}
A^T(B^{-1}A^{-1} + I)^T 
& = A^T(A^{-T}B^{-T} + I) \\
& = A^TA^{-T}B^{-T} + A^T \\
& = B^{-T} + A^T \\
& = (B^T)^{-1} + A^T = A^{-1} + B
\end{aligned}
$$

\end{proof}


\section{Problem 7}
Solution: 

Let $A = [\alpha_1^T, \alpha_2^T, \alpha_3^T, \alpha_4^T]$, and $x = (x_1, x_2, x_3, x_4)^T$. Solve linear system $Ax = \alpha^T$ and we obtain $x = (\frac{5}{4}, \frac{1}{4}, -\frac{1}{4}, -\frac{1}{4})^T$.

\section{Problem 8}
Solution: 

Build matrix $A = [\alpha_1, \alpha_2, \alpha_3]$. vectors $\alpha_1, \alpha_2$ and $\alpha_3$ are linearly independent if and only if $A$ is of full rank. Hence, we only need to check the rank of $A$ by Gauss elimination.

$$
A = \begin{pmatrix}
1 & 0 & 1 \\
1 & 2 & 3 \\
1 & 5 & 6
\end{pmatrix} \xrightarrow {Gauss \quad elimination}
\begin{pmatrix}
1 & 0 & 1 \\
0 & 2 & 2 \\
0 & 0 & 0
\end{pmatrix}
$$

This implies that $Rank(A) = 2 < 3$. In conclusion, $\alpha_1, \alpha_2$ and $\alpha_3$ are NOT linearly independent.

\section{Problem 9}
\begin{proof}
First step, we'd like to show $\delta_1 + \delta_2$, $\delta_2 + \delta_3$ and $\delta_3 + \delta_1$ are linearly independent based on $\delta_1$, $\delta_2$ and $\delta_3$ are linearly independent. 

Recall the definition of linearly dependence: 
 $\delta_1$, $\delta_2$ and $\delta_3$ are linearly independent if 
 $$
 \alpha_1\delta_1 + \alpha_2\delta_2 + \alpha_3\delta_3 = 0 \iff \alpha_1 = \alpha_2 = \alpha_3 = 0.
 $$
 
Hence suppose $\beta_1(\delta_1+\delta_2) + \beta_2(\delta_2+\delta_3) + \beta_3(\delta_3+\delta_1) = 0$, we only need to show that $\beta_1 = \beta_2 = \beta_3 = 0$. 

Rearrange the left side of it and obtain $(\beta_1 + \beta_3)\delta_1 + (\beta_1 + \beta_2)\delta_2 + (\beta_2 + \beta_3)\delta_3 = 0$. From the independence of $\delta_1$, $\delta_2$ and $\delta_3$, we have 
$$
\beta_1 + \beta_3 = 0, \beta_1 + \beta_2 = 0, \beta_2 + \beta_3 = 0.
$$
which implies $\beta_1 = \beta_2 = \beta_3 = 0$. 

For the opposite direction of this claim, use the similar idea above and we only need to show that 

$$
\eta_1\delta_1 + \eta_2\delta_2 + \eta_3\delta_3 = 0 \Rightarrow \eta_1 = \eta_2 = \eta_3 = 0
$$

based on the linear independence of $\delta_1 + \delta_2$, $\delta_2 + \delta_3$ and $\delta_3 + \delta_1$.

Suppose $\eta_1\delta_1 + \eta_2\delta_2 + \eta_3\delta_3 = 0$. And construct coefficient $\gamma_1 = \frac{1}{2}(\eta_2 + \eta_1 - \eta_3)$, $\gamma_1 = \frac{1}{2}(\eta_2 - \eta_1 + \eta_3)$ and $\gamma_1 = \frac{1}{2}(\eta_3 - \eta_2 + \eta_1)$. It is easy to verify that

$$
\gamma_1(\delta_1 + \delta_2) + \gamma_2(\delta_2+\delta_3) + \gamma_3(\delta_3+\delta_1) = \eta_1\delta_1 + \eta_2\delta_2 + \eta_3\delta_3 = 0.
$$

By the independence of $\delta_1 + \delta_2$, $\delta_2 + \delta_3$ and $\delta_3 + \delta_1$, we obtain $\gamma_1 = \gamma_2 = \gamma_3 = 0$, which implies $\eta_1 = \eta_2 = \eta_3 = 0$.

Hence, the claim is proved.
\end{proof}



\section{Problem 10}
Solution:
\begin{enumerate}
\item
$$
A \xrightarrow {Gauss \quad elimination}
\begin{pmatrix}
1 & 2 & 3 & 4 & 5 \\
0 & 0 & 1 & 2 & 3 \\
0 & 0 & 0 & 0 & 4 \\
0 & 0 & 0 & 0 & 0
\end{pmatrix}
$$

From this we know that $Rank(A) = 3$.

\item
$$
A \xrightarrow {Gauss \quad elimination}
\begin{pmatrix}
3 & 2 & -1 & -3 & -2 \\
0 & -7/3 & 11/3 & 3 & -5/3 \\
0 & 0 & 0 & 1 & 2 
\end{pmatrix}
$$

From this we know that $Rank(A) = 3$.
\end{enumerate}


\end{document}