\documentclass[12pt]{article}
\usepackage[margin=1in]{geometry} 
\usepackage{amsmath,amsthm,amssymb,amsfonts}
\usepackage{placeins}
\usepackage{mathtools, eucal}
 

 
\begin{document}
 
%\renewcommand{\qedsymbol}{\filledbox}
%Good resources for looking up how to do stuff:
%Binary operators: http://www.access2science.com/latex/Binary.html
%General help: http://en.wikibooks.org/wiki/LaTeX/Mathematics
%Or just google stuff
 

\textbf{Issued: Sept. 7, 2017 } \qquad 
\textbf{ISE/MA/OR 505 HW3} \qquad 
\textbf{Due: Sept. 14, 2017}
\vspace{3mm}
\hrule


\section*{\underline{Exercises}}

\begin{enumerate}
\item  
(10 points)

Consider the following problem. 

\begin{equation*}
\begin{aligned}
\text{Maximize} \quad & 3x_1 -2x_2 + 4|x_3| \\
\text{subject\  to} \quad & -x_1 + 2x_2 \leqslant -5 \\
& 3x_2 - x_3 \geqslant 6 \\
& x_1, x_2 \geqslant 0
\end{aligned}
\end{equation*}

To avoid the potential problems of not including the quadratic requirement in converting a free variable to the standard-form LP, you are asked to convert this problem into a pair of standard-form linear programs that may collectively provide the correct solution to the original problem using the simplex method. Please remember to discuss how your answer works and any potential drawbacks of this approach. 


\vspace{5mm}

The following problems are in \textbf{Textbook Chapter 2}

\item (10 points) 2.1
\item (15 points) 2.2
\item (10 points) 2.3
\item (10 points) 2.4
\item (10 points) 2.5
\item (10 points) 2.6
\item (15 points) 2.8
\item (10 points) 2.9

\end{enumerate}

\section*{\underline{Reading Assignment}}

Textbook Chapter 3



\end{document}