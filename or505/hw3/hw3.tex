\documentclass[12pt]{article}
\usepackage[margin=1in]{geometry} 
\usepackage{amsmath,amsthm,amssymb,amsfonts}
\usepackage{placeins}
\usepackage{mathtools, eucal}
 

 
\begin{document}
 
%\renewcommand{\qedsymbol}{\filledbox}
%Good resources for looking up how to do stuff:
%Binary operators: http://www.access2science.com/latex/Binary.html
%General help: http://en.wikibooks.org/wiki/LaTeX/Mathematics
%Or just google stuff
 
\title{Homework 2 Solutions}
\author{Zheming Gao}
\maketitle


\section*{Problem 1}

Since there is an absolute value in the objective function and $x_3$ is unrestricted, we may divide the feasible domain into two parts: $x_3 \geqslant 0$ and $x_3 < 0$.

For $x_3 \geqslant 0$, $|x_3| = x_3$. Hence, we may construct a LP problem:

\begin{equation*}
\begin{aligned}
\text{Manximize} \quad & 3x_1 - 2x_2 + 4 x_3 \\
\text{subject\  to} \quad & -x_1 + 2 x_2 \leqslant -5 \\
& 3 x_2 - x_3 \geqslant 6 \\
& x_1, x_2, x_3 \geqslant 0
\end{aligned}
\end{equation*}

Similarly, for $x_3 < 0$,$|x_3| = -x_3$, and we can also construct a LP problem:

\begin{equation*}
\begin{aligned}
\text{Manximize} \quad & 3x_1 - 2x_2 - 4 x_3 \\
\text{subject\  to} \quad & -x_1 + 2 x_2 \leqslant -5 \\
& 3 x_2 - x_3 \geqslant 6 \\
& x_1, x_2 \geqslant 0, x_3 \leqslant 0.
\end{aligned}
\end{equation*}

Then convert those two LP problems above into standard forms:

\begin{equation}\label{L1}
\begin{aligned}
\text{Minimize} \quad & -3x_1 + 2x_2 - 4 x_3 \\
\text{subject\  to} \quad & -x_1 + 2 x_2 + \xi_1 = -5 \\
& 3 x_2 - x_3 - \xi_2 = 6 \\
& x_1, x_2, x_3, \xi_1, \xi_2 \geqslant 0
\end{aligned}
\end{equation}

and 

\begin{equation}\label{L2}
\begin{aligned}
\text{Minimize} \quad & -3x_1 + 2x_2 - 4 x_3 \\
\text{subject\  to} \quad & -x_1 + 2 x_2 + \xi_1 = -5 \\
& 3 x_2 + x_3 - \xi_2 = 6 \\
& x_1, x_2, x_3, \xi_1, \xi_2 \geqslant 0
\end{aligned}
\end{equation}

Take the optimal solution as the one that solves (\ref{L1}) or (\ref{L2}) with a smaller optimal value.



\section*{Problem 2.1}

Solutions: 

\begin{enumerate}
\item
We need to show that if the feasible domain is bounded, then the LP problem has a bounded optimal value.

\begin{proof}

Consider the standard form of a LP problem.

\begin{equation*}
\begin{aligned}
\text{Minimize} \quad & c^Tx \\
\text{subject\  to} \quad & Ax = b \\
 & x \geqslant 0
\end{aligned}
\end{equation*}

where $A \in \mathbb{R}^{m\times n}$, $x\in \mathbb{R}^n$, $b \in \mathbb{R}^m$.

Denote its feasible domain as $P := \{x \in \mathbb{R}^n | Ax = b, x \geqslant 0 \}$. If P is bounded, then $\exists M \geqslant 0$ such that $||x|| \leqslant M$ for all $x \in P$.

Recall Cauchy-Schwartz inequality, $\forall x, y \in \mathbb{R}^n$, 
$$
|x^Ty| \leqslant ||x||\cdot||y||
$$

Hence, we have 
$$
c^Tx \geqslant - ||c||\cdot ||x|| = ||c||(-||x||) \geqslant ||c||\cdot (-M).
$$

Since $c$ is a constant vector, $M$ exists as a constant, we know $c^Tx$ has a lower bound, which proves that the LP problem is bounded.

\end{proof}

\item

Next we give a counterexample to show that the opposite direction is not ture. 

Consider the following LP problem:

\begin{equation}\label{L2}
\begin{aligned}
\text{Minimize} \quad & 2x_1 + x_2 \\
\text{subject\  to} \quad & x_1 + x_2 \geqslant 1 \\
& x_1, x_2 \geqslant 0
\end{aligned}
\end{equation}

It is obvious that the feasible domain is not bounded but the problem is bounded.

\end{enumerate}


\section*{Problem 2.3}

\begin{proof}

To prove $H$ is affine, take $x, y$ from $H$ arbitrarily and we need to show the affine combination of $x$ and $y$ is also in $H$. This is true, since for any $\alpha_1, \alpha_2 \in \mathbb{R}$ that satisfy $\alpha_1 + \alpha_2  =1$, 

$$
a^T(\alpha_1 x + \alpha_2 y) = \alpha_1 a^T x + \alpha_2 a^T y = \alpha_1\beta + \alpha_2 \beta = \beta.
$$

which shows $\alpha_1 x + \alpha_2 y$ is in $H$.

For convexity, it follows from the fact that $H$ is affine because the convex combination of two points is a special case of their affine combination.

\end{proof}


\section*{Problem 2.4}

\begin{enumerate}
\item

\begin{proof}

We want to show $\forall x, y \in \cap_{i = 1}^ p C_i$, the convex combination of $x, y$ is also in it.

Since $x, y \in \cap_{i = 1}^ p C_i$, we know $x, y \in C_i$, $\forall i = 1,\dots, p$. with the fact that $C_i$ is convex, for any $\alpha \in (0, 1)$, $\alpha x + (1-\alpha) y \in C_i$ holds for each index $i$. Hence, $\alpha x + (1-\alpha) y \in \cap_{i=1} ^ p C_i$. 

The claim is then proved.

\end{proof}

\item
$\cup_{i=1}^p$ may not be convex. A counterexample is to let $C_1 = \{(x, y)| x = 0, y\in\mathbb{R}\}$, and $C_1 = \{(x, y)| y = 0, x\in\mathbb{R}\}$. It is clear that $C_1, C_2$ are convex and they are $x$ and $y$ axis in $\mathbb{R}^2$. However, $C_1\cup C_2$ is not convex. 

Let $a = (1, 0) \in C_2$, $b = (0, 1)\in C_1$. $a, b \in C_1\cup C_2$, but $\frac{1}{2} a + \frac{1}{2}b = (1/2, 1/2) \notin C_1\cup C_2$.

\end{enumerate}


\section*{Problem 2.5}

(It will be easy to use the results from the problems we just solved. But it is fine if you use other methods to prove this claim.)

\begin{proof}

Let $A \in \mathbb{R}^{m\times n}$ be $\begin{bmatrix}
a_1^T \\
\vdots \\
a_m^T
\end{bmatrix}
$, where $a_i^T$ is the ith row of $A$. 
and $b = [b_1, \cdots, b_m]^T$. Then the feasible domain $P$ is equivalent to 

$$
\cap_{i=1}^m \{ x\in\mathbb{R}^n | a_i^T x = b_i  \} \cap \{x\in\mathbb{R}^n | x \geqslant 0\}.
$$

Let $P_i : = \{ x\in\mathbb{R}^n | a_i^T x = b_i  \}$ and each $P_i$ is a hyperplane. Also, it is obvious that $\{x\in\mathbb{R}^n | x \geqslant 0\}$ is convex(use the definition and easy to prove). Use the results from 2.3 and 2.4, and we know that the $P_i$ is convex and intersection of convex sets is also convex. Hence, $P$ is convex.

\end{proof}

\section*{Problem 2.6}

\begin{proof}

Consider a LP problem in standard form. 

\begin{equation*}
\begin{aligned}
\text{Minimize} \quad & c^Tx \\
\text{subject\  to} \quad & Ax = b \\
 & x \geqslant 0
\end{aligned}
\end{equation*}

where $A \in \mathbb{R}^{m\times n}$, $x\in \mathbb{R}^n$, $b \in \mathbb{R}^m$.

Denote its feasible domain as $P := \{x \in \mathbb{R}^n | Ax = b, x \geqslant 0 \}$. Let the supporting hyperplane $H$ of feasible domain $P$ be the following,

$$
H: = \left\{  x\in\mathbb{R}^n| -c^Tx = \beta \right\}.
$$

and $\forall x\in P$, $-c^Tx \leqslant \beta$. since $H\cap P \neq \phi$, take any $x^* \in H\cap P$, we have $c^Tx^* =  -\beta$. Hence, $\forall x\in P$, $c^Tx \geqslant c^Tx^* = -\beta$, which proves that $x^*$ is an optimal solution to the LP problem.

\end{proof}










\end{document}