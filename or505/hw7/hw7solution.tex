\documentclass[12pt]{article}
\usepackage[margin=1in]{geometry} 
\usepackage{amsmath,amsthm,amssymb,amsfonts}
\usepackage{enumitem}
\usepackage{placeins}
\usepackage{mathtools, eucal}
\usepackage{graphicx}
\usepackage{color}
\usepackage{subcaption}
 

 
\begin{document}
 
%\renewcommand{\qedsymbol}{\filledbox}
%Good resources for looking up how to do stuff:
%Binary operators: http://www.access2science.com/latex/Binary.html
%General help: http://en.wikibooks.org/wiki/LaTeX/Mathematics
%Or just google stuff
 
\title{Homework 7 Solutions}
\author{Zheming Gao}
\maketitle

\section*{Problem 1 (4.7)}

\begin{proof}

Suppose that the dual is feasible and bounded. Then, it has a finite optimum. By strong duality theorem, the dual of dual, which is the primal, also has a finite optimum. But this is a contradiction to the infeasibility of primal problem.

\end{proof}

\section*{Problem 2 (4.8)}





\section*{Problem 4 (4.11)}


\begin{proof}

If $Ax = b, x\geqslant 0$ has a solution $x_0$ and $A^Tw \geqslant 0$, then $x_0^TA^Tw \geqslant$. This implies that $b^Tw\geqslant 0$.

Conversely, we need to show that if $Ax = b, x\geqslant 0$ has no solution, then $b^Tw <0$ as $A^w \geqslant 0$. Indeed, this is true due to Farkas Lemma.

\end{proof}

\section*{Problem 5 (4.13)}

\begin{proof}

If $Ax \leqslant b, x\geqslant 0$ has a solution $x_0$ and $A^Tw \geqslant 0, w\geqslant 0$, then $x_0^TA^T\leqslant b^T$. Multiply $w$ on both sides and we have 

$$
0 = x_0^TA^Tw \leqslant b^Tw.
$$

which is $b^Tw \geqslant 0$.


Conversely, we need to show that if $b^Tw\geqslant 0$ when $A^Tw \geqslant 0, w\geqslant 0$, then $Ax \leqslant b, x\geqslant 0$ has a solution. Consider the following primal dual problem,

\begin{equation}\label{4.13primal}
\begin{aligned}
\text{Min} \quad  & 0^Tx \\
\text{Subject to} \quad & Ax \leqslant b \\
& x\geqslant 0
\end{aligned}
\end{equation}

\begin{equation}\label{4.13dual}
\begin{aligned}
\text{Max} \quad  & b^Ty \\
\text{Subject to} \quad & A^Ty \leqslant 0 \\
& y\leqslant 0
\end{aligned}
\end{equation}

(\ref{4.13dual}) is equivalent to (\ref{4.13dual2})

\begin{equation}\label{4.13dual2}
\begin{aligned}
\text{Max} \quad  & -b^Tw \\
\text{Subject to} \quad & A^Tw \geqslant 0 \\
& w\geqslant 0
\end{aligned}
\end{equation}

In (\ref{4.13dual2}), it is obvious that $w=0$ is a feasible solution. What's more, it is also an optimal solution due to the assumption that $b^Tw \geqslant 0$ when $A^Tw \geqslant 0, w\geqslant 0$. Hence, $\max -b^Tw = 0$. By strong duality theorem, we know (\ref{4.13primal}) is also feasible.


\end{proof}

\section*{Problem 6 (4.18)}

Standard form of the primal:

$$
\begin{aligned}
\text{Minimize} \qquad & 2x_1 + x_2 - x_3 &  \\
\text{subject\  to} \qquad & x_1 + 2x_2 + x_3 + a_1 & = 8\\
 & -x_1 + x_2 -2x_3 + a_2 & = 4 \\
 & x_1, x_2, x_3, a_1, a_2 & \geqslant 0
\end{aligned}
$$ 



From point $x = [x_1, x_2, x_3, a_1, a_2]^T = [0, 0, 0, 8, 4]^T$, we know that $B = [A_4, A_5]$ and $N = [A_1, A_2, A_3]$. Compute reduced cost $r = c_N^T - c_B^TB^{-1}N = [2, 1, -1]$.

Note that $r_3$ is negative, so $x_3$ enter the basis. Construct $$
M^{-1} = \begin{bmatrix}
B^{-1} & -B^{-1}N \\
 0 & I
\end{bmatrix} = \begin{bmatrix}
1 & 0 & -1 & -2 & -1 \\
0 & 1 & 1 & -1 & 2 \\
0 & 0 & 1 & 0 & 0 \\
0 & 0 & 0 & 1 & 0 \\
0 & 0 & 0 & 0 & 1
\end{bmatrix}
$$ 

to figure out $\textbf d^3 = [-1, 1, 1, 0, 0]^T$  and the step length $\alpha_3 = 8$ . Hence, $x_{\text{new}} = x + \alpha_3\textbf{d}^3 = [0, 0, 8, 0, 20]^T$. 

Next step:

From point $x = [x_1, x_2, x_3, x_4, x_5]^T = [0, 0, 8, 0, 20]^T$, we know that $B = [A_3, A_5]$ and $N = [A_1, A_2, A_4]$. Compute reduced cost $r = c_N^T - c_B^TB^{-1}N = [3,3,1]^T \geqslant 0$. Hence, this is the optimal solution. The optimal solution is $x^* = [0, 0, 8, 0, 20]^T$ and the optimal value $z^* = c^Tx^* = -8 $. The optimal dual is $w^{*T} = c^T_BB^{-1} = (-1, 0)$.

\begin{enumerate}

\item [(a)]

$x_2$ is not a basic variable, so $c_B^TB^{-1}$ doesn't change. Also, to check optimality, consider 

$$
r^T = c_N^T - c_B^TB^{-1}N = [2, c_2, 0] - [-1, -2, -1] = [3, c_2 + 2, 1]. 
$$

where $c_2$ now is $6$. Hence, $r^T \geqslant 0$ so that the solution remains optimal.

\item [(b)]

$A_2$ changed, but $B$ and $c_B^T$ don't change. So,

$$
r^T = c_N^T - c_B^TB^{-1}N = [2, 1, 0] - [-1, 0] \begin{bmatrix}
1 & 0 \\ 2 & 1
\end{bmatrix} \begin{bmatrix}
1 & a_{12} & 1 \\
-1 & 1 & 0
\end{bmatrix} = [3, 1+ a_{12}, 1]. 
$$

where $a_{12} = 0.25$, $1+ a_{12} = 1.25 >0$. Hence, the solution remains optimal.

\item [(c)]

If we add new constraint $x_2 + x_3 = 3$, then substitute $x_2$ with $x_2 = 3-x_3$ and we have a LP problem,

$$
\begin{aligned}
\text{Minimize} \qquad & 2x_1  - 2x_3 + 3 &  \\
\text{subject\  to} \qquad & x_1 - x_3 & \leqslant 2\\
 & -x_1 - 3x_3 & \leqslant 1 \\
 & x_3 & \leqslant 3 \\
 & x_1, x_3 & \geqslant 0
\end{aligned}
$$ 

Solve it and we get infinitely many optimal solutions. $x^* = [x_1, x_3]^T$ lies on the segment ($x_1-x_3$ plane) between $[2, 0]^T$ and $[5,3]^T$. Hence the optimal value is $z^* = 1$.

\item [(d)]




\end{enumerate}



\end{document}