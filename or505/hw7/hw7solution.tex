\documentclass[12pt]{article}
\usepackage[margin=1in]{geometry} 
\usepackage{amsmath,amsthm,amssymb,amsfonts}
\usepackage{enumitem}
\usepackage{placeins}
\usepackage{mathtools, eucal}
\usepackage{graphicx}
\usepackage{color}
\usepackage{subcaption}
 

 
\begin{document}
 
%\renewcommand{\qedsymbol}{\filledbox}
%Good resources for looking up how to do stuff:
%Binary operators: http://www.access2science.com/latex/Binary.html
%General help: http://en.wikibooks.org/wiki/LaTeX/Mathematics
%Or just google stuff
 
\title{Homework 7 Solutions}
\author{Zheming Gao}
\maketitle

\section*{Problem 1 (4.7)}

\begin{proof}

Suppose that the dual is feasible and bounded. Then, it has a finite optimum. By strong duality theorem, the dual of dual, which is the primal, also has a finite optimum. But this is a contradiction to the infeasibility of primal problem.

\end{proof}

\section*{Problem 2 (4.8)}






\section*{Problem 4 (4.11)}


\begin{proof}

If $Ax = b, x\geqslant 0$ has a solution $x_0$ and $A^Tw \geqslant 0$, then $x_0^TA^Tw \geqslant$. This implies that $b^Tw\geqslant 0$.

Conversely, we need to show that if $Ax = b, x\geqslant 0$ has no solution, then $b^Tw <0$ as $A^w \geqslant 0$. Indeed, this is true due to Farkas Lemma.

\end{proof}

\section*{Problem 5 (4.13)}

\begin{proof}

If $Ax \leqslant b, x\geqslant 0$ has a solution $x_0$ and $A^Tw \geqslant 0, w\geqslant 0$, then $x_0^TA^T\leqslant b^T$. Multiply $w$ on both sides and we have 

$$
0 = x_0^TA^Tw \leqslant b^Tw.
$$

which is $b^Tw \geqslant 0$.


Conversely, we need to show that if $b^Tw\geqslant 0$ when $A^Tw \geqslant 0, w\geqslant 0$, then $Ax \leqslant b, x\geqslant 0$ has a solution. Consider the following primal dual problem,

\begin{equation}\label{4.13primal}
\begin{aligned}
\text{Min} \quad  & 0^Tx \\
\text{Subject to} \quad & Ax \leqslant b \\
& x\geqslant 0
\end{aligned}
\end{equation}

\begin{equation}\label{4.13dual}
\begin{aligned}
\text{Max} \quad  & b^Ty \\
\text{Subject to} \quad & A^Ty \leqslant 0 \\
& y\leqslant 0
\end{aligned}
\end{equation}

(\ref{4.13dual}) is equivalent to (\ref{4.13dual2})

\begin{equation}\label{4.13dual2}
\begin{aligned}
\text{Max} \quad  & -b^Tw \\
\text{Subject to} \quad & A^Tw \geqslant 0 \\
& w\geqslant 0
\end{aligned}
\end{equation}

In (\ref{4.13dual2}), it is obvious that $w=0$ is a feasible solution. What's more, it is also an optimal solution due to the assumption that $b^Tw \geqslant 0$ when $A^Tw \geqslant 0, w\geqslant 0$. Hence, $\max -b^Tw = 0$. By strong duality theorem, we know (\ref{4.13primal}) is also feasible.


\end{proof}




\end{document}