\documentclass[12pt]{article}
\usepackage[margin=1in]{geometry} 
\usepackage{amsmath,amsthm,amssymb,amsfonts, }
\usepackage{enumitem}
\usepackage{placeins}
\usepackage{mathtools, eucal}
\usepackage{listings}  
\usepackage{graphicx}
\usepackage{color}
\usepackage{subcaption}

\usepackage{amsmath,amsgen,amstext,amsbsy,amsopn,amsfonts,amsthm}
% standard AMS packages for symbols, formats, etc.
\usepackage{graphicx,tabularx,array,geometry,float}
%\usepackage{hyperref}
% packages for web interface, graphics, etc,

 
\newtheorem{remark}{Remark}
 
\begin{document}
 
%\renewcommand{\qedsymbol}{\filledbox}
%Good resources for looking up how to do stuff:
%Binary operators: http://www.access2science.com/latex/Binary.html
%General help: http://en.wikibooks.org/wiki/LaTeX/Mathematics
%Or just google stuff
 
\title{Homework 9 Solutions}
\author{Fangroup}
\maketitle

\section*{Problem 1}

\begin{enumerate}
\item [(a)]
We list the key results at each iteration.
\begin{table}[h]
\small
\centering
    \begin{tabular}{|c|c|c|c|c|}
      \hline
      Iteration $k$ & $d_y^k=-X_kr^k$ & Step length $\alpha_k$ & $x^{k+1}$ & $c^Tx^k - c^Tx^{k+1}$ \\\hline
      0 & $(-0.0682,0.0227,0.0227)$ & $14.52$ & $(0.0025,0.6650,0.3325)$ & $0.0825$ \\\hline
      1 & $(-0.8333,0.0021,0.0021)\times 10^{-3}$ & $1.1880\times 10^3$ & $(0.0000,0.6666,0.3333)$ & $8.2500\times 10^{-4}$ \\\hline
      2 & $(-0.8333,0.0000,0.0000)\times 10^{-5}$ & $1.1880\times 10^5$ & $(0.0000,0.6667,0.3333)$ & $8.2500\times 10^{-6}$ \\\hline
    \end{tabular}
\end{table}

Results:

\begin{lstlisting}
%Solution:
solu =
    0.0000
    0.6667
    0.3333

%Number of iterations
itnum =
     6
%Objective value
obj =
    1.6667

\end{lstlisting}

\item [(b)]

\begin{table}[h]
\footnotesize
    \begin{tabular}{|c|c|c|c|c|}
      \hline
      Iteration $k$ & 0 & 1  & 2 \\
      \hline
      $d_w^k$ & $( -0.5714, 1.6327)$ & $(0.0221, 0.0223)$ & $(0.0888, 0.1853)\times 10^{-3}$  \\
      \hline
      $d_s^k$ & $(-1.0612, -0.4898, -2.2041)$ & $(-0.0444, -0.0665, -0.0003)$ & $(-0.0965, -0.0077, -0.2741)\times10^{-3}$  \\
      \hline
      $\beta_k$ & $1.3475$ & $5.0597$ &   \\
      \hline
      $w^k$ & $(0,0)$ & $(-0.7700, 2.2000)$ & $(-0.6582, 2.3131)$\\
      \hline
      $s^k$ & $(2,1,3)$ & $( 0.5700, 0.3400, 0.0300)$ & $ (0.3452, 0.0034, 0.0287)$ \\
      \hline
    \end{tabular}
\end{table}

Results:

\begin{lstlisting}
%Solution:
solu_w =
   -0.6582
    2.3131
solu_s =
    0.3452
    0.0034
    0.0287

%Number of iterations
itnum =
     2
%Objective value
obj =
    1.6548    
\end{lstlisting}

\end{enumerate}

\section*{Problem 2}
It is on HW8.

\section*{Problem 3}

\begin{itemize}
  \item[(a)] At the vertex $x=(0,0,0.5,0.5)$, the reduced costs are $r_1=-1$ and $r_2=0$. So the moving direction is $d_1=(1,0,-0.5,-0.5)$.
  \item[(b)] $x=(0.01,0.01,0.49,0.49)$. Use Karmarkar's algorithm. $d^k_y=(0.0075,-0.0025,-0.0025,-0.0025)$. Since the transformation between $x$ and $y$ is nonlinear, we don't have a moving ``direction'' in $x-$space.
  \item[(c)] Use the primal affine scaling algorithm. Moving direction is $(0.9998,-0.0002,-0.4998,-0.4998)\times 10^{-4}$.
  \item[(d)] Set $\mu=0.01$. The central force is $(0.0098,0.098,-0.0098,-0.0098)$.

  Moving direction is $(0.0198,0.0098,-0.0148,-0.0148)$.
  \item[(e)] Compare the moving directions given by different algorithms. Think about how they are defined. Consider their effects on the performance of the algorithms. Think about the meaning of $\mu$.
\end{itemize}

\section*{Problem 4}

\begin{itemize}
  \item[(a)] Dual problem is
  \begin{equation*}
    \begin{array}{rl}
      \max & w_2+1 \\
      \text{s.t.} & w_2\leq -1 \\
      & w_2 \leq 0 \\
      & w_1+w_2\leq 0 \\
      & -w_1+w_2\leq 0
    \end{array}
  \end{equation*}
  \item[(b)] Obviously.
  \item[(c)] At $w=(1,-2)$, we find $s=(1,2,1,3)$. $d^k_w=(-0.4848,0.6061)$. $d^k_s=(-0.6061,-0.6061,-0.1212,-1.0909)$.
  \item[(d)] This moving direction is pointing to one of the dual optimal solutions.
  \item[(e)] Set $\mu=0.01$. Centering force is $(-0.5152,1.3939)$. The moving direction is $(-47.9697,59.2121)$.
  \item[(f)] Whether the direction in (e) is better than in (c) depends on the choice of $\mu$. Please try different $\mu$. Some directions fail to point to an optimal solution.
\end{itemize}



\section*{Problem 5}

\begin{itemize}
\item[(a)]Based on the representation of $x=y+q$, we may convert the original LP problem to the following LP problem in terms of the new variable $y$:
\begin{align*}
\min & \quad c^Ty+c^Tq\\
\text{s.t.} & \quad Ay=b-Aq\\
& \quad y\geq 0.
\end{align*}
Subtracting the constant $c^Tq$ in the objective, it becomes the following standard form LP problem:
\begin{align*}
\min & \quad c^Ty\\
\text{s.t.} & \quad Ay=b-Aq\\
& \quad y\geq 0.
\end{align*}
\item[(b)] For the above standard form LP problem, its dual problem is given by
\begin{align*}
\min & \quad (b-Aq)^Tw\\
\text{s.t.} & \quad A^Tw\leq c.
\end{align*}
When $q=0$, the above dual problem becomes
\begin{align*}
\min & \quad b^Tw\\
\text{s.t.} & \quad A^Tw\leq c,
\end{align*}
which is actually a regular dual problem for the original LP problem with $q=0$.
\end{itemize}

\end{document}

