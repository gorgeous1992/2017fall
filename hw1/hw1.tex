\documentclass[12pt]{article}
 \usepackage[margin=1in]{geometry} 
\usepackage{amsmath,amsthm,amssymb,amsfonts}
\usepackage{mathtools}
 

 
\begin{document}
 
%\renewcommand{\qedsymbol}{\filledbox}
%Good resources for looking up how to do stuff:
%Binary operators: http://www.access2science.com/latex/Binary.html
%General help: http://en.wikibooks.org/wiki/LaTeX/Mathematics
%Or just google stuff
 
\title{Homework 1 Solutions}
\author{Zheming Gao}
\maketitle

\section{Problem 1}

Solutions: 
\begin{enumerate}
\item 
\begin{equation*}
det
\begin{pmatrix}
a^2 & ab \\
ab & b^2
\end{pmatrix}
= a^2b^2 - abab = 0.
\end{equation*}

\item
\begin{equation*}
det
\begin{pmatrix}
\cos\alpha & -\sin\alpha \\
\sin\alpha & \cos\alpha
\end{pmatrix}
= \cos^2\alpha + \sin^2\alpha = 1.
\end{equation*}

\item
\begin{equation*}
det
\begin{pmatrix}
1 & x & x \\
x & 2 & x \\
x & x & 3
\end{pmatrix}
= 1\times 2 \times 3 + x^3 + x^3 - 2x^2 - x^2 - 3x^2 = 
2x^3 - 6x^2 + 6.
\end{equation*}

\item
Similar to the problem above, 
\begin{equation*}
\begin{aligned}
det
\begin{pmatrix}
1 & 1 & 1 \\
a & b & c \\
a^2 & b^2 & c^2
\end{pmatrix}
& = bc^2 + a^2c + ab^2 - a^2b - ac^2 - b^2c \\
& = ab(b-a) + ac(a-c) + bc(c-b).
\end{aligned}
\end{equation*}

\item
Change the order of columns. And apply the operations on matrix blocks:
\begin{equation*}
\begin{aligned}
det
\begin{pmatrix}
1 & 0 & 2 & a\\
2 & 0 & b & 0 \\
3 & c & 4 & 5 \\
d & 0 & 0 & 0
\end{pmatrix}
& = 
- det
\begin{pmatrix}
0 & 2 & a & 1\\
 0 & b & 0 & 2 \\
c & 4 & 5 & 3\\
0 & 0 & 0 & d
\end{pmatrix}
 \\
& = 
-det 
\begin{pmatrix}
0 & 2 & a\\
0 & b & 0 \\
c & 4 & 5
\end{pmatrix}
\cdot d \\
& = abcd.
\end{aligned}
\end{equation*}

\item
Same idea as the problem above. Note that it won't change the determinant of a matrix if we add one row multiplied by a constant number on another row. 

suppose $a\neq 0$. Then, add $\frac{1}{a}$row1 on row2. The determinant doesn't change.
\begin{equation*}
det
\begin{pmatrix}
a & 1 & 0 & 0\\
-1 & b & 1 & 0 \\
0 & -1 & c & 1 \\
0 & 0 & -1 & d
\end{pmatrix}
= 
det
\begin{pmatrix}
a & 1 & 0 & 0\\
0 & b + \frac{1}{a} & 1 & 0 \\
0 & -1 & c & 1 \\
0 & 0 & -1 & d
\end{pmatrix}
\end{equation*}

Then change orders of rows, and change orders of columns. 
\begin{equation*}
\begin{aligned}
det
\begin{pmatrix}
a & 1 & 0 & 0\\
0 & b + \frac{1}{a} & 1 & 0 \\
0 & -1 & c & 1 \\
0 & 0 & -1 & d
\end{pmatrix}
& = 
- det
\begin{pmatrix}
0 & b + \frac{1}{a} & 1 & 0 \\
0 & -1 & c & 1 \\
0 & 0 & -1 & d \\
a & 1 & 0 & 0
\end{pmatrix} \\
& = 
det
\begin{pmatrix}
b + \frac{1}{a} & 1 & 0 & 0 \\
-1 & c & 1 & 0 \\
0 & -1 & d & 0\\
1 & 0 & 0 & a
\end{pmatrix} \\
& = 
det 
\begin{pmatrix}
b + \frac{1}{a} & 1 & 0\\
-1 & c & 1 \\
0 & -1 & d
\end{pmatrix} 
\cdot a \\
& = 1 + ab + ad + cd + abcd.
\end{aligned}
\end{equation*}

\end{enumerate}


\section{Problem 2}

Solution:
All linear systems can be formed as $Ax = b$. 
\begin{enumerate}
\item
In this problem, $x \in \mathbb R^4$, $A$ is a 4 by 4 matrix and $b$ is a 4 by 1 vector.

\begin{equation*}
[A|b] = \begin{bmatrix}
2 & -1/2 & -1/2 & 0 & 0 \\
-1/2 & 2 & 0 & -1/2 & 3 \\
-1/2 & 0 & 2 & -1/2 & 3 \\
0 & -1/2 & -1/2 & 2 & 0 
\end{bmatrix} \xrightarrow {Gauss \ \ elimination}
\begin{bmatrix}
1 & -1/4 & -1/4 & 0 & 0 \\
0 & 1 & -1/15 & -4/15 & 8/5 \\
0 & 0 & 1 & -2/7 & 12/7 \\
0 & 0 & 0 & 1 & 1 
\end{bmatrix}
\end{equation*}

Thus, the solution to the linear system is $x = (1, 2, 2, 1)^T$.

\item
\begin{equation*}
[A|b] = \begin{bmatrix}
2 & 3 & 5 & 1 & 3 \\
3 & 4 & 2 & 3 & -2 \\
1 & 2 & 28 & -1 & 8 \\
7 & 9 & 1 & 8 & 0 
\end{bmatrix} \xrightarrow {Gauss \ \ elimination}
\begin{bmatrix}
1 & 3/2 & 5/2 & 1/2 & 3/2 \\
0 & 1 & 11 & -3 & 13 \\
0 & 0 & 0 & 0 & 0 \\
0 & 0 & 0 & 0 & 1 
\end{bmatrix}
\end{equation*}

We see $[A|b]$ is not in full rank so there is no solution to the linear system.
\end{enumerate}

\section{Problem 3}
$A = P\Lambda Q$, $A^2 = P\Lambda QP\Lambda Q$. Since $QP = I$, $A^2 = P \lambda^2 Q$. Hence, $A^k = P\Lambda ^k Q$. 

It is enough to calculate what $\Lambda^k$. Since $\Lambda$ is a diagonal matrix, 
\begin{itemize}
\item k is a odd number, 

$\Lambda ^k = \Lambda$. Then 
\begin{equation*}
A^k = P\Lambda Q = \begin{pmatrix}
7 & -12 \\
6 & -7
\end{pmatrix}.
\end{equation*}
\item k is an even number, 

$\Lambda ^k = I$. Then 
\begin{equation*}
A^k = PIQ = PQ = I.
\end{equation*}

\end{itemize}

\section{Problem 4}

The linear system can be formed as $Ax = b$, where
\begin{equation*}
A = \begin{bmatrix}
1 & 1 & 1\\
0 & 2 & 2\\
1 & -1 & 0
\end{bmatrix}, \ \ 
b = \begin{bmatrix}
1 \\
1 \\
2
\end{bmatrix}
\end{equation*}

It is clear that $A$ is invertible.(You may check that A is of full rank)
To find the inverse of $A$, i.e. $A^{-1}$, there is one method by Gauss elimination

\begin{equation*}
[A|I] = \begin{bmatrix}
1 & 1 & 1 & 1 & 0 & 0 \\
0 & 2 & 2 & 0 & 1 & 0 \\
1 & -1 & 0 & 0 & 0 & 1
\end{bmatrix} \xrightarrow {Gauss \ \ elimination}
\begin{bmatrix}
1 & 0 & 0 & 1 & -1/2 & 0 \\
0 & 1 & 0 & 1 & -1/2 & -1 \\
0 & 0 & 1 & -1 & 1 & 1 
\end{bmatrix} = [I|A^{-1}]
\end{equation*}

The solution is $x = A^{-1}b = (1/2, -3/2, 2)^T$.


\section{Problem 5}





\begin{enumerate}
\item

\end{enumerate}


 
\begin{proof}
Proof goes here. Repeat as needed
\end{proof}

\end{document}