\documentclass[12pt]{article}
 \usepackage[margin=1in]{geometry} 
\usepackage{amsmath,amsthm,amssymb,amsfonts}
\usepackage{mathtools, eucal}
\usepackage{xparse}
 
\DeclarePairedDelimiter\floor{\lfloor}{\rfloor}

 
\begin{document}
 
%\renewcommand{\qedsymbol}{\filledbox}
%Good resources for looking up how to do stuff:
%Binary operators: http://www.access2science.com/latex/Binary.html
%General help: http://en.wikibooks.org/wiki/LaTeX/Mathematics
%Or just google stuff
 
\title{MA 515 Homework 1}
\author{Zheming Gao}
\maketitle

\section*{Problem 1}

\begin{proof}

To show that $(X, d)$ is a metric space, we need to verify that $d(x, y)$ is well defined as a distance. Obviously, $\forall x, y in X$, $d(x, x) = 0$ and $d(x, y) = d(y, x)$ due to the properties of absolute value. Hence, we only need to verify the triangle inequality holds.

Let $a = \sqrt{|x - y|}$ and $b = \sqrt{|y - z|}$. Hence, 
$a$ and $b$ are non-negative, which promise the following relations,
$$
(a + b)^2 - (a^2 + b^2) = 2ab \geqslant 0.
$$

Therefore, we have
$$
d(x, y) + d(y, z) = a + b \geqslant \sqrt{a^2 + b^2} = \sqrt{|x - y|+|y - z|} \geqslant \sqrt{|x - z|} = d(x, z).
$$

which proves the triangle inequality. In conclusion, $(X, d)$ is a metric space.

\end{proof}


\section*{Problem 2}

Yes. $(X, d)$ is a metric space.

\begin{proof}

We will use the result from problem 1 to show this. Still, with the properties of absolute value and integration, we know that $\forall f, g \in X$, $d(f, f) = 0$ and $d(f, g) = d(g, f)$. Then, we only need to prove the triangle inequality. $\forall f, g, h \in X$, use the linearity of integration, triangle inequality for absolute value and result from problem 1, we have

$$
\begin{aligned}
d(f, g) + d(g, h) & = \int_a^b |g(t) - f(t)| + \sqrt{|g(t) - f(t)|} dt + \int_a^b |h(t) - g(t)| + \sqrt{|h(t) - g(t)|} dt \\
& =  \int_a^b |g(t) - f(t)| + |h(t) - g(t)| + \sqrt{|g(t) - f(t)|} + \sqrt{|h(t) - g(t)|} dt \\
& \geqslant \int_a^b |h(t) - f(t)| + \sqrt{|h(t) - f(t)|} dt = d(f, h).
\end{aligned}
$$

In conclusion, $(X, d)$ is a metric space.

\end{proof}

\section*{Problem 3}

We will prove (b) first and use it to prove (a).

\begin{enumerate}
\item (b)
\begin{proof}
Since $(X,d)$ is a metric space, so $\forall x, y ,z \in X$ we apply triangle inequality on $d(x, z)$ and obtain $d(x, z) \leqslant d(x, y) + d(y, z)$, from where we know that $d(x, z)- d(y, z) \leqslant d(x, y)$. Similarly, from $d(y, z) \leqslant d(y, x) + d(x, z)$, we know $d(y, z)- d(x, z) \leqslant d(x, y)$.

In conclusion, $\forall x, y, z \in X$, 
$$
|d(x, z)- d(y, z)| \leqslant d(x, y).
$$

\end{proof}

\item (a)

\begin{proof}

Now we use (b) to show (a). Rewrite left hand side and use triangle inequality,
$$
\begin{aligned}
LHS = |d(x, y) - d(z, w)| & = |d(x, y) - d(y, z) + d(y, z) - d(z, w)| \\
& \leqslant |d(x, y) - d(y, z)| + |d(y, z) - d(z, w)| \\
& = |d(y, x) - d(y, z)| + |d(z, y) - d(z, w)| \\
& \leqslant d(x, z) + d(y, w)
\end{aligned}
$$

This proves the inequality in $(a)$.

\end{proof}

\end{enumerate}

\section*{Problem 4}

\begin{enumerate}
\item (a) 

Example:

We might let $x = \{x_n\}_{n\in\mathbb{N}}$ and $x_n = 1/\sqrt{n}$. This satisfies $x_n \rightarrow 0$ as $n \rightarrow +\infty$, but doesn't satisfy $x \in \ell^2$ since

$$
\sum_{n = 1}^{+\infty} |x_n|^2 = \sum_{n = 1}^{+\infty} \frac{1}{n} = +\infty.
$$

\item (b)

Example: 

Use the same $x = \{x_n\}_{n\in\mathbb{N}} = \{1/\sqrt{n}\}$. $x \notin \ell^2$, but $x \in \ell^3$ since 

$$
\sum_{n = 1}^{+\infty} |x_n|^3 = \sum_{n = 1}^{+\infty}\left(\frac{1}{n}\right)^{3/2} < +\infty.
$$

\end{enumerate}


\section*{Problem 5}

\begin{proof}
We will prove this by contradiction. Let $x: = \{x_n\}_{n=1}^{+\infty} \in \ell^p$, so $x$ satisfies $\sum_{n=1}^{+\infty} |x_n|^p < +\infty$. $(p > 1)$

Suppose that $x$ is not a bounded sequence, i.e., for any $M > 0$, $\exists N$ such that $\forall n > N$, $|x_n| > M$. We might let $M = 1$, then there exists $N_0$ such that 
$\forall n > N_0$,

$$
\sum_{n = 1}^{+\infty} |x_n|^p \geqslant \sum_{n = N_0 + 1}^{+\infty} |x_n|^p > \sum_{n = N_0 + 1}^{+\infty} M^p > +\infty.
$$

which contradicts to the fact that $x \in \ell^p$. Hence, $x$ is a bounded sequence. Equivalently, $x$ is in $\ell^{+\infty}$.

\end{proof}

\section*{Problem 6}

\begin{enumerate}
\item (a)

\begin{proof}
Need to show that $d_f(x, y)$ is a metric. First of all, $\forall x \in X$, since $(X, d)$ is a metric space, we have

$$
d_f(x, x) = f \circ d(x, x) = f(0) = 0.
$$

Also, $\forall x, y \in X$, 

$$
d_f(x, y) = f\circ d(x, y) = f \circ d(y, x) = d_f(y, x).
$$

For triangle inequality, $\forall x, y, z \in X$, use supper-additivity of $f$ and obtain

$$
\begin{aligned}
d_f(x, y) + d_f(y, z) & = f\circ d(x, y) + f\circ d(y, z) \\
& \geqslant f\circ (d(x, y) + d(y, z)) \geqslant f\circ d(x, z) = d_f(x, z).
\end{aligned}
$$

In conclusion, $(X, d_f)$ is a metric space.

\end{proof}

\item (b)

\begin{proof}

Let $f: \mathbb{R}_+ \rightarrow \mathbb{R}_+$. $f(x) = x/(1 + x)$. It is clear that 
\begin{enumerate}
\item [(i)]
$$
f(0) = 0\ \text{and}\  f(s) > 0 \ \text{for all}\  s>0,
$$

\item [(ii)]
$f$ has supper-additivity, because $\forall x, y \geqslant 0$, 

$$
\begin{aligned}
f(x + y) - f(x) - f(y) & = \frac{x+y}{1+x+y} - \frac{x}{1+x} - \frac{y}{1+y} \\
& = \left( \frac{x}{1+x+y} - \frac{x}{1+x} \right) + \left( \frac{y}{1+x+y} - \frac{y}{1+y} \right) \leqslant 0.
\end{aligned}
$$

With the result in part (a), $d_f(x, y) = \tilde{d}(x, y)$ is a metric.

\end{enumerate}

\end{proof}

\end{enumerate}

\section*{Problem 7}
\begin{enumerate}
\item (a)
\begin{proof}
To prove $d$ is a metric, we need to verify three conditions. First of all, $\forall x\in X$, $d(x, x) = d_1(x_1, x_1) + d_2(x_2, x_2) = 0$. 

Also, $\forall x, y \in X$, 
$$
d(x, y) = d_1(x_1, y_1) + d_2(x_2, y_2) = d_1(y_1, x_1) + d_2(y_2, x_2) = d(y, x).
$$

For triangle inequality, $\forall x, y, z \in X$, we have 

$$
\begin{aligned}
d(x, y) + d(y, z) & =  d_1(x_1, y_1) + d_2(x_2, y_2) +  d_1(y_1, z_1) + d_2(y_2, z_2) \\
& = [d_1(x_1, y_1) + d_1(y_1, z_1)] + [d_2(x_2, y_2) +   d_2(y_2, z_2)] \\
& \geqslant d_1(x_1, z_1) + d_2 (x_2, z_2) = d(x, z).
\end{aligned}
$$

Hence, $d$ is a metric.

\end{proof}


\item (b)

\begin{proof}

Similarly to part (a), it is only need to show the triangle inequality for $\tilde d$.

$\forall x, y \in X$, let $\tilde d (x, y) = ||x - y||$. Actually, $||\cdot||$ is the Euclidean norm defined on $\mathbb R^2$. We would like to show that the triangle inequality holds for $||\cdot||$. And it is enough to show that $\forall a, b \in X$, $||a+b|| \leqslant ||a|| + ||b||$. Use Cauchy-Schwartz inequality,

$$
\begin{aligned}
||a + b||^2 = (a + b)^T(a + b) & = ||a||^2 + ||b||^2 + 2a^Tb \\
& \leqslant ||a||^2 + ||b||^2 + 2 ||a||||b|| = (||a||+||b||)^2
\end{aligned}
$$

Hence, $\tilde d $ is a metric.

\end{proof}
\end{enumerate}


\section*{Problem 8}

\begin{proof}

$\forall x, y \in X$ and $K \subset X, K \neq \phi$, 
consider $d(x, y) + d_K(x)$.

$$
\begin{aligned}
d(x, y) + d_K(x) & = d(x, y) + \inf_{w \in K} d(x, w) \\
& = \inf_{w \in K} \left\{  d(x, y) + d(x, w) \right\} \\
& \geqslant \inf_{w\in K} d(y, w) = d_K(y).
\end{aligned}
$$

Hence, $d_K(y) - d_K(x) \leqslant d(x, y)$.

Similarly, consider $d(x, y) + d_K(y)$ and with similar argument, we will get $d(x, y) \geqslant d_K(y) - d_K(x)$.

In all, $d_K$ is 1-Lipschitz continuous. i.e., 
$$
|d_K(y) - d_K(x)| \leqslant d(x, y) \quad \forall x, y \in X.
$$

\end{proof}

\end{document}