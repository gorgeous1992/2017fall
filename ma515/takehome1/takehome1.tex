\documentclass[12pt]{article}
 \usepackage[margin=1in]{geometry} 
\usepackage{amsmath,amsthm,amssymb,amsfonts}
\usepackage{mathtools, eucal}
\usepackage{xparse}
 
\DeclarePairedDelimiter\floor{\lfloor}{\rfloor}

 
\begin{document}
 
%\renewcommand{\qedsymbol}{\filledbox}
%Good resources for looking up how to do stuff:
%Binary operators: http://www.access2science.com/latex/Binary.html
%General help: http://en.wikibooks.org/wiki/LaTeX/Mathematics
%Or just google stuff
 
\title{MA 515 Prerequisites Test Solutions}
\author{Zheming Gao}
\maketitle

\section*{Problem 1}

\begin{proof}

For all $\epsilon > 0$, we need to find a $N$ such that $\forall n > N$, $|x_n - 2| < \epsilon$.

We may let $N = \lfloor{2/\epsilon - 1} \rfloor + 1$. Then, for all $n > N$, 

$$
|\frac{2n}{n+1} - 2| - \epsilon = \frac{2}{n+1} - \epsilon < \frac{2}{\lfloor{2/\epsilon - 1} \rfloor + 2} < \frac{2}{2/\epsilon} = \epsilon.
$$

In conclusion, the sequence $\{x_n\}$ converges to $2$.

\end{proof}


\section*{Problem 2}
During the proof, we will need the linearity of limit. i.e.
$\alpha \in \mathbb{R}, \{x_n\} \rightarrow 3$ and $\{y_n\} \rightarrow 5$ , then
\begin{itemize}
\item 
$$
\lim_{n\rightarrow \infty}\alpha x_n = 3\alpha
$$

\item
$$
\lim_{n\rightarrow \infty} x_n + y_n = 3 + 5 = 8.
$$
\end{itemize}

\begin{proof}
We need to show that $\forall \epsilon > 0, \exists N$, such that $\forall n > N$, $|x_ny_n - 15| <  \epsilon$.

Since both $\{x_n\}$ and $\{y_n\}$ are convergent, we know,$\forall \epsilon_1 > 0, \exists N_1, N_2$, such that for all $m > N_1, n > N_2$, 
 
$$
|x_m - 3| < \sqrt{\epsilon_1}, \quad |y_n - 5| < \sqrt{\epsilon_1}.
$$

We may let $N_3 = \max{N_1, N_2}$. Then for all $n > N_3$, 

$$
|(x_n - 3)(y_n - 5)| < \epsilon_1.
$$

In last inequality, use triangle inequality and obtain the left-hand side: 

$$
\begin{aligned}
LHS = |x_ny_n - 3y_n - 5x_n + 30 -15| & \geqslant |x_ny_n - 15| - |3y_n + 5x_n - 30| \\
& \geqslant |x_ny_n - 15| - |3(y_n - 5)| - |5(x_n - 3)| \\
& \geqslant |x_ny_n - 15| - 8\sqrt{\epsilon_1}
\end{aligned}
$$

Hence, we have the following,
$$
|x_ny_n - 15| \leqslant 8\sqrt{\epsilon_1} + \epsilon_1 \quad \rightarrow 0 
$$

as $\epsilon$ goes to 0.

Then the claim of $\lim_{n \rightarrow \infty} x_ny_n = 15$ is proved.



\end{proof}

\section*{Problem 3}

\begin{proof}

From $\{x_n\}$ is a Cauchy sequence we know that $\forall \epsilon > 0, \exists N$ such that $\forall n, m > N$, $|x_n - x_m| < \epsilon$. We may let $m = N + 1$ and then $\forall n > N, |x_n - x_{N+1}| < \epsilon$, which implies $|x_n| < \epsilon + |x_{N+1}|$.

For all $\epsilon > 0$, we may let $K_\epsilon = \max\{|x_1|, |x_2|, \dots, |x_N|, |x_{N+1} + \epsilon|\} $. Then $\forall n \in \mathbb{N}$, $|x_n| < K_\epsilon $. Here $K_\epsilon < +\infty$.

Since $\epsilon$ is arbitrary chosen and we let it go to 0. Then $K_\epsilon \rightarrow K < +\infty$ and $\{x_n\}$ is bounded by $K$.

Hence, Cauchy sequence is bounded.


\end{proof}


\section*{Problem 4}

Recall the definition of continuous. A function $f$: $\mathbb{R} \rightarrow \mathbb{R}$ is continuous at $c$ if $f$ is defined at $c$ and satisfies that $\forall \epsilon > 0$, there exists $\delta > 0$ such that $\forall |x_n - c| < \delta$, $|f(x_n) - f(c)| < \epsilon$.

\begin{proof}
Since the sequence $\{x_n\}$ converges to $c$, we know that $\forall \epsilon_0 > 0$, $\exists N_0$ such that $|x_n - c| \epsilon_0$. With the fact that $f$ is continuous at $c$, $\forall \epsilon > 0$, there exists $\delta > 0$. And for this $\delta$, there exists $N_\epsilon$, such that $\forall n > N_\epsilon$, 
$$
|f(x_n) - f(c)| < \epsilon.
$$

Rewrite above, we have $\forall \epsilon > 0, \exists N_\epsilon$, such that 
$$
\forall n > N_\epsilon, \quad |f(x_n) - f(c)| < \epsilon.
$$

which proves the equivalent statement of $\lim_{n\rightarrow \infty} f(x_n) = f(c)$.

\end{proof}


\section*{Problem 5}
$\forall x = (x_1, x_2), y = (y_1, y_2) \in \mathbb{R}^2$, let metric $d_1(x, y) = ||x - y||$, where $||x|| = \sqrt{x_1^2 + x_2^2}$ is the norm on $X$. If we show $(X, d_1)$ is a metric space, then we can use this to show the triangle inequality for metric $d$.

$\forall x\in X$, let $\bar x = (2x_1, 3x_2)$, and it is clear that $d(x, y) = d_1(\bar x, \bar y)$. Then $\forall x, y, z \in X$, 
$$
d(x,y) + d(y, z) = d_1(\bar x, \bar y) + d_1(\bar y, \bar z) \geqslant d_1(\bar x, \bar z) = d(x, z).
$$

Moreover, from the definition we can see that $d$ satisfies 
\begin{enumerate}
\item $d(x, x) = 0$
\item $d(x, y) = d(y, x)$.
\end{enumerate}

Hence, $(X, d)$ is a metric space.

In the following we only need to show that $(X, d_1)$ is a metric space.
\begin{proof}
We see that $d_1$ satisfies that $d(x, x) = 0$ and $d(x, y) = d(y, x)$. So the only thing we need to show is the triangle inequality of $d_1$. i.e., $\forall x, y, z \in X$, 
$$
||x- y|| + ||y - z|| \geqslant ||x - z||.
$$

We may let $x - y = a,  y - z = b$ and $ x - z = a + b$. Then use Cauchy-Schwartz inequality,

$$
\begin{aligned}
||a + b||^2 = (a + b)^T(a + b) & = ||a||^2 + ||b||^2 + 2a^Tb \\
& \leqslant ||a||^2 + ||b||^2 + 2 ||a||||b|| = (||a||+||b||)^2
\end{aligned}
$$

Take the square root on both sides and obtain $||a+b|| \leqslant ||a|| + ||b||$, which proves the triangle inequality of metric $d_1$.

\end{proof}

\section*{Problem 6}
\begin{proof}

Obviously, $\{x_n\}$ is a Cauchy sequence. Recall that in problem 3, we have already shown that $\{x_n\}$ is bounded. Use Bolzano-Weierstrass theorem and we know that there exists a  convergent subsequence $\{x_{n_k}\}$. Let's say, the subsequence converges to $c$. i.e., $\forall \epsilon > 0$, $\exists N_1$, such that $\forall n_k > N_1$, $|x_{n_k} - c| < \epsilon/2$. what's more, since $\{x_n\}$ is Cauchy, we know $\exists N_2$, such that $\forall m, n > N_2$, $|x_m - x_n|< \epsilon/2$.

Combine those two conditions and take $N_3 = \max\{N_1, N_2\}$, we see that $\forall n, n_k > N_3$,
$$
|x_n - c| = |x_n - x_{n_k} + x_{x_k} - c| \leqslant |x_n - x_{n_k}| + |x_{x_k} - c| < \epsilon/2 + \epsilon/2 = \epsilon.
$$

which proves the claim that $\{x_n\}$ is convergent.

\end{proof}



\end{document}