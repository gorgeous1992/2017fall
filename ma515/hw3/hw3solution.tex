\documentclass[12pt]{article}
 \usepackage[margin=1in]{geometry} 
\usepackage{amsmath,amsthm,amssymb,amsfonts}
\usepackage{color}
\usepackage{mathtools, eucal}
\usepackage{xparse}
\usepackage{romannum}
 
\DeclarePairedDelimiter\floor{\lfloor}{\rfloor}

 
\begin{document}
 
%\renewcommand{\qedsymbol}{\filledbox}
%Good resources for looking up how to do stuff:
%Binary operators: http://www.access2science.com/latex/Binary.html
%General help: http://en.wikibooks.org/wiki/LaTeX/Mathematics
%Or just google stuff
 
\title{MA 515 Homework 3}
\author{Zheming Gao}
\maketitle

\section*{Problem 1}

\begin{enumerate}
\item [(a)]

It is not a norm. It violates 







\vspace{60mm}

\item [(b)]

\begin{proof}

If $E$ is closed in X, then $X \setminus E$ is open in $X$. Use the result from (a), it is equivalent to say that there exists an open set $\mathcal{O} \subset Y$ such that $X\setminus E = \mathcal{O} \cap X$. And this implies,

$$
E = X\setminus (O\cap X) = X \setminus \mathcal{O} \cup \phi = X \setminus \mathcal{O} = ( X \setminus \mathcal{O}) \cap Y = (Y\setminus \mathcal{O}) \cap X.
$$

Let $F = Y\setminus \mathcal{O}$ and $F$ is closed in $Y$.

\end{proof}

\end{enumerate}


\section*{Problem 9}

\begin{proof}

let $g(x) = \frac{1}{4} e^{f(x)}$. We need to show that $g: \mathbb{R} \rightarrow \mathbb{R}$ is a contraction map. If so, then there exists a unique $x_0\in\mathbb{R}$ such that $g(x_0) = x_0$, which is the unique solution to the equation.

For any $x\in \mathbb{R}$, $f(x) \in [0, 1]$. And also, let $h(z) = e^z, z\in [0, 1]$ and the derivative of $h$ is in $[1, e]$. Thus, there exists $0<c<1$, such that

$$
|g(x) - g(y)| = \frac{1}{4}|e^{f(x)} - e^{f(y)}| \leqslant \frac{1}{4}\max |h^\prime|\cdot |f(x) - f(y)| \leqslant \frac{e}{4}c|x-y|.
$$

Let $c^\prime = \frac{e}{4}c$ and note that $c^\prime \in (0, 1)$. Hence, $g$ is a contraction map.


\end{proof}

\section*{Problem 10}
\begin{enumerate}
\item
\begin{proof}

Apply triangle inequality and the property of norm, we have
$$
\begin{aligned}
||\lambda_1e_1 + \lambda_2e_2|| & \leqslant |\lambda_1|\cdot||e_1|| + |\lambda_2|\cdot||e_2|| \leqslant \max \{ ||e_1||, ||e_2|| \}(|\lambda_1| + |\lambda_2|).
\end{aligned}
$$

holds for all $\lambda_1, \lambda_2 \in \mathbb{R}$. Let $\beta_2 = \max \{ ||e_1||, ||e_2|| \}$ and it is clear that $\beta_2 > 0$. and the claim was proved.

\end{proof}

\item

\begin{proof}

Use the same idea in last proof. Let $\beta_n = \max_{1\leqslant i \leqslant n} \{ ||e_i|| \}$ and it is clear that $\beta_n > 0$. 

$$
\begin{aligned}
||\sum_{i=1}^n \lambda_i\cdot e_i|| & \leqslant \sum_{i=1}^n|\lambda_i|\cdot||e_i|| \leqslant \max_{1\leqslant i \leqslant n} \{ ||e_i|| \} \sum_{i=1}^n|\lambda_i| = \beta_n \sum_{i=1}^n|\lambda_i|
\end{aligned}, \quad \forall \lambda_1, \lambda_2 \in \mathbb{R}.
$$

The claim was proved.


\end{proof}


\end{enumerate}




\section*{Problem 2}

\begin{enumerate}
\item [(a)]

\begin{proof}

We want to show $\overline{U\cup V} \subset \overline U \cup \overline V$. Take $x \in \overline U$, by definition of closure, $\forall \epsilon >0$, $\exists y \in U\subset (U\cup V)$ such that  $d(x, y) < \epsilon$. This implies that $x \in \overline{U\cup V}$. Hence, $\overline U \subset \overline{U\cup V}$. Similarly, $\overline V \subset \overline{U\cup V}$. In all, $\overline U \cup \overline V \subset \overline{U\cup V}$.

For the other direction, we prove it by contradiction. Suppose $\exists y \in \overline{U \cup V}$ such that $y \notin \overline U \cup \overline V$. This is to say that $y\notin \overline U$ and $y \notin \overline V$. Then there exists $\epsilon > 0$, such that, $d(x, y) \geqslant \epsilon$ and $d(x, z) \geqslant \epsilon$ for all $x\in U, z \in V$. i.e, $\forall w \in U\cup V$, $d(w, y) > \epsilon$, which is $y\notin \overline{U \cup V}$ (contradiction).

In conclusion, $\overline{U\cup V} = \overline U \cup \overline V$.

\end{proof}


\item [(b)]

Generally, $\overline{U\cap V} = \overline U \cap \overline V$ is not true. Here is a counterexample. Let $X = \mathbb R$, $U = (-1, 0)$ and $V = (0, 1)$. Then $\overline U = [-1, 0]$ and $\overline V = [0, 1]$, and it implies that $\overline U \cap \overline V = \{ 0 \} $. However, $\overline {U\cap V} = \phi$ since $U\cap V = \phi$.

\item [(c)]

\begin{proof}

$\forall x \in \overline U$, there exists a sequence $\{x_n\}$ in $U$ such that $x_n \rightarrow x$. Since $U \subset V$, then $x_n$ also converges in $\overline V$. Hence, $x \in \overline V$. This proves that $\overline U \subset \overline V$.

\end{proof}

\end{enumerate}

\section*{Problem 3}
\begin{proof} 
\ 
\begin{enumerate}
\item [Step 1]

We would like to show that if $x\in K$, then $d_K(x) = 0$. By definition, $\forall x\in K$, $d_K(x) = \inf_{w\in K} d(x, w)$. Since $d(x, w) \geqslant 0$ and $d(x, w) = 0$ when $w = x$. Hence, $d_K (x) = \inf_{w\in K} d(x, w) = d(x, x) = 0$.

\item [Step 2]

Conversely, let's prove it by contradiction. When $d_K(x) = 0$, suppose $x \notin K$, which is to say that $x\in K^c$. Since $K$ is closed, $K^c$ is open. Then $\exists \delta > 0 $ such that $B(x, \delta) \in K^c$, which implies $B(x, \delta) \cap = \phi$.  Thus, $\forall w \in K$, $d(x, w) > \delta > 0$, and it implies $\inf_{w\in K} d(x, w) \geqslant \delta > 0$. This leads to a contradiction to the assumption.

\end{enumerate}

\end{proof}

\section*{Problem 4}

\begin{proof}

Consider $f$ as a continuous map from $X$ to $Y$. $\forall F \subset Y$ closed, we have $Y \setminus F \subset Y$ is open in $Y$. Then $f^{-1} (Y\setminus F) = f^{-1}(Y)\setminus f^{-1}(F) = X\setminus f^{-1}(F)$ is open in $X$. This implies that $f^{-1}(F)$ is closed in $X$.

Conversely, $\forall E \subset Y$ open subset in $Y$, $Y\setminus E$ is closed in $Y$. 

$$
f^{-1}(E) = f^{-1}(Y\setminus (Y\setminus E)) = f^{-1}(Y) \setminus f^{-1}(Y\setminus E) = X \setminus f^{-1}(Y\setminus E).
$$

Since $f^{-1}(Y\setminus E)$ is closed in $X$, $f^{-1}(E)$ is open in $X$. Hence, the preimage of any open set is also open and this satisfies the definition of continuous function. Thus, $f$ is continuous. 

\end{proof}


\section*{Problem 5}

\begin{proof}

For one direction, if $f$ is not continuous, then by definition, there exists $\epsilon_0 > 0$, for each $n$, $\exists x_n \in X$, such that $|x_n - x| < 1/n$, but $\sigma (f(x_n), f(x)) \geqslant \epsilon_0$. And this implies a contradiction, for then $x_n \rightarrow x$ but $f(x_n)$ doesn't converge to $f(x)$.

For the other direction, we know $f$ is continuous, i.e.,  $\forall \epsilon > 0$, $\exists \delta > 0$, such that $\sigma (f(y) - f(x)) < \epsilon$ holds for all $d(x, y) < \delta$. With $x_n \rightarrow x$, there exists integer $N_\delta > 0$, such that $d(x_n , x) < \delta$ holds for all $n > N_\delta$.

Hence, $\forall \epsilon > 0$, $\exists N_\delta > 0$ such that $\sigma (f(x_n) - f(x)) < \epsilon$ holds for all $n > N_\delta$. And this is equivalent to $f(x_n) \rightarrow f(x)$.

\end{proof}


\section*{Problem 6}

\begin{proof}

Prove by contradiction. Suppose a Cauchy sequence $\{x_n\}$ does not converges to $x$, though it has subsequence $\{x_{n_k}\}$ that converges to $x\in X$. Then $\exists \epsilon > 0$, for any $N > 0$, $d(x_n, x) \geqslant \epsilon$ for some $n > N$. Since $\{x_{n_k}\}$ converges to $x$, there exists $N_1 > 0$, such that $d(x_{n_k}, x) < \epsilon/2$, $\forall n_k > N_1$. Also, $\{x_n\}$ is Cauchy and it leads to $\exists N_2 > 0$, such that $d(x_m, x_n) <\epsilon/2$, $\forall m, n > N_2$. 

Taking $N = \max\{N_1, N_2\}$ and for some $n, n_k > N$, we use triangle inequality and get,

$$
\epsilon \leqslant d(x_n, x) \leqslant d(x_n, x_{n_k}) + d(x_{n_k}, x) < \epsilon/2 + \epsilon/2 = \epsilon.
$$

which is obviously a contradiction.

\end{proof}


\section*{Problem 7}

\begin{proof}

To prove $l^2$ is complete, it is enough to show that all Cauchy sequences converge in $l^2$. We will show this step by step. Pick a Cauchy sequence $\{x_n\}_{n\in\mathbb N}\subset l^2$, we construct a point $x$, and we need to show that $x\in l^2$ and $x_n\rightarrow x$.

\begin{enumerate}
\item

Take a Cauchy sequence $\{x^n\}_{n\in\mathbb N} \subset l^2$, where the $i^{th}$ element is $x^i := (x_1^i, x_2^i, \dots)$. Since $\{x^n\}$ is Cauchy, $\forall \epsilon > 0$, $\exists N > 0$, such that 

$$
|x_i^n - x_i^m| <= \left( \sum^{\infty}_{j=1} |x^n_j - x^m_j|^2\right)^{1/2} < \epsilon. \quad \forall n,m > N, \forall i
$$

Hence, $\{x_i^n\}_{n\in\mathbb N}$ is Cauchy in $\mathbb R$, which is a complete metric space. Hence, $x_i^n \rightarrow x_i$, $\forall i \geqslant 1$.

Consider first $N$ entries of the point in $l^2$. Denote $y^i_N : = (x_1^i, x_2^i, \dots, x_N^i)$ and $y_N = (x_1, x_2, \dots, x_N)$. And it is clear that $\{y^n_N\}_{n\in\mathbb N}$ converges to $y_N$. This is obvious since this is in the finite dimensional case. $\forall 1\leqslant i\leqslant N$, there exists $N_i$ such that $|x_i^n - x_i| < \epsilon / \sqrt{N}$, $\forall n > N_i$. Take $\overline N = \max_{1\leqslant i \leqslant N}\{N_i\}$, we have 

$$
\left( \sum^{N}_{j=1} |x^n_j - x^m_j|^2\right)^{1/2} < \epsilon.
$$

Notice that $N$ is arbitrarily chosen, so we may take limit of $N$ on both sides of the inequality above and get

$$
d(x^n, x) = \left( \sum^{\infty}_{j=1} |x^n_j - x^m_j|^2\right)^{1/2} \leqslant \epsilon.
$$

And this proves that $x^n\rightarrow x$ in $l^2$. 

\item

Next we need to show that $x \in l^2$. This is proved by the following

$$
\sum_{i = 1}^\infty |x_i|^2 = \sum_{i = 1}^\infty \lim_{n\rightarrow\infty}|x_i^n|^2 = \lim_{n\rightarrow\infty}\sum_{i = 1}^\infty |x_i^n|^2 < +\infty.
$$

Infinite summation and limit can exchange due to dominate convergence theorem(DCT). Hence, $x\in l^2$.

\end{enumerate}

In conclusion, $\{x^n\}$ converges to $x$ in $l^2$, which proves that $l^2$ is complete.

\end{proof}


\section*{Problem 8}

\begin{proof}

Take a Cauchy sequence $\{x_n\}$ on $(X, \tilde{d})$. Then $\forall \epsilon > 0$, $\exists N > 0$, such that $\tilde d(x_m, x_n) < \epsilon $, for all $m, n > N$. Since $\tilde d(x, y) = d(x, y)/(1+d(x, y))$, we know $d(x, y) < \epsilon / (1-\epsilon)$. It is clear that $\epsilon / (1-\epsilon) \rightarrow 0$ as $\epsilon \rightarrow 0$. Hence, $\{x_n\}$ is also a Cauchy sequence on $(X, d)$. With the fact that $(X, d)$ is complete, $\{x_n\}$ is convergent and it leads to the completeness of $(x, \tilde{d})$.


\end{proof}

\section*{Problem 9}

\begin{proof}

Take a Cauchy sequence $\{z_n\}$ in $(Y, d_Y)$. Since $(Y, d_Y)$ is complete, we know that $\forall \epsilon > 0$, $\exists N$ such that $d_Y(z_m, z_n) < \epsilon$, $\forall m,n > N$. 

With the fact that $T$ is isometric, $T$ is automatically injective, otherwise two distinct points will be mapped to the same point and it leads to a contradiction since the distance between two same points is $0$. Also, $T(X) = Y$, i.e. T is surjective. Hence, $T$ is a bijection from $X$ to $Y$. Therefore there exists a inverse mapping of $T$ such that $\forall y\in Y$, $\exists x\in X$ such that $T^{-1}(y) = x$. Apply this on the Cauchy sequence,  $\exists \{x_n\}$ such that $T(x_n) = z_n$, for all $n\in\mathbb{N}$. Hence, 
$$
d_Y(z_m, z_n)<\epsilon \Leftrightarrow d_Y(T(x_m), T(x_n))<\epsilon \Leftrightarrow d_X(x_m, x_n)< \epsilon.
$$

And this implies that $\{x_n\}$ is Cauchy on $(X, d_X)$ and so it converges. Let $x_n \rightarrow x\in(x, d_X)$, i.e, $\forall \epsilon > 0$, $\exists N$ such that $d_X(x_n, x) = d_Y(T(x_n), T(x))< \epsilon$ for all $n > N$. Hence, $\{T(x_n)\} = \{z_n\}$ converges to $T(x)$ in $(Y, d_Y)$. Thus, any Cauchy sequence on $(Y, d_Y)$ converges in itself, which leads to the completeness of $(Y, d_Y)$.

\end{proof}


\section*{Problem 10}

{
\color{red} I didn't figure the answer by myself. Actually, I discussed this problem with other students in class and got the ideas from them. 
}
\begin{proof}

For $l^\infty$, the unit ball $B(0, 1)$ is not totally bounded. Let $\epsilon = 1/4$ and $x_1, x_2, \dots, x_n \in B(0,1)$ be arbitrarily finitely many points. Then define $v = \{v_n\}_{n\in\mathbb N}$ as:

$$
v_i \in (-1, 1) \setminus (x_i^{(i)} - 1/4, x_i^{(i)} + 1/4).
$$

where $x_i^{(i)}$ is ith entry of $x_i$, $i \geqslant 1$. 

Hence, 

$$
\sup_{n \geqslant 1} |v_n -x_j^{(n)}| \geqslant |v_j - x_j^{(j)}| \geqslant 1/4 \qquad \forall j \in \mathbb N.
$$

Hence, $v \notin \cup _{i=1}^n B(x_i, 1/4)$.

In conclusion, $B(0,1)$ is not totally bounded.

\end{proof}





\section*{Problem 11}

\begin{proof}

\begin{enumerate}
\item Step 1

We want to show that if $E$ is totally bounded, then $\bar E$ is totally bounded. By definition, $\forall \epsilon >0$, there exists finitely many points $a_1, a_2, \dots, a_{N_\epsilon} \in E$ such that $E \subset \cup_{i=1}^{N_\epsilon} B(a_i, \epsilon/2)$. Since $E$ is dense in $\bar E$, $\forall x \in \bar E$, $\exists y \in E$ such that $d(x, y)<\epsilon/2$. This is equivalent that $\exists a \in \{a_1, a_2, \dots, a_{N_\epsilon}\}$ such that $\exists y \in B(a, \epsilon/2)$, and $d(x, y) < \epsilon/2$. With triangle inequality, 

$$
d(a, x) \leqslant d(a, y) + d(y, x) < \epsilon/2 + \epsilon/2 = \epsilon.
$$

Hence, $x$ is in $B(a, \epsilon)$, which leads to $x \in \cup_{i=1}^{N_\epsilon} B(a_i, \epsilon/2)$. Since $x$ is arbitrarily chosen from $\bar E$, we conclude that $\bar E \subset \cup_{i=1}^{N_\epsilon} B(a_i, \epsilon/2)$. i.e, $\bar E$ is also totally bounded.

\item Step 2

Conversely, let $\bar E$ is totally bounded, then $\forall \epsilon > 0$, $\exists a_1, \dots, a_{N_\epsilon}$ such that $\bar E \subset \cup_{i=1}^{N_\epsilon} B(a_i, \epsilon/2)$. With the fact that $E$ is dense in $\bar E$, for any $a_i$, there exists $b_i \in E$ such that $d(b_i, a_i)<\epsilon$. Hence, by triangle inequality, $\forall y \in B(a_i, \epsilon)$, 

$$
d(y, b_i) \leqslant d(y, a_i) + d(a_i, b_i) < \epsilon/2 + \epsilon/2 = \epsilon.
$$

Hence, $y \in B(b_i, \epsilon)$, which implies $\bar E \in \cup_{i=1}^{N_\epsilon} B(b_i, \epsilon)$. Since $E\subset \bar E$ and $b_i \in E, \forall i$, we conclude that $E \subset \cup_{i=1}^{N_\epsilon} B(b_i, \epsilon)$, which is equivalent to that $E$ is totally bounded.


\end{enumerate}

\end{proof}

\section*{Problem 12}

\begin{proof}

Since $K$ is compact and $f$ is continuous, then $f(K)$ is compact on $\mathbb R$. This implies that $f(K)$ is closed and bounded and the existence of $\max{f(K)}$ in $f(K)$. Let $f_{\text{max}} = \sup{f(K)} = \max {f(K)}$. Since $f_{\text{max}} \in f(K)$, then there exist $x_{\text{max}}\in K$ such that $f_{\text{max}} = f(x_{\text{max}})$. 

\end{proof}

\section*{Problem 13}

\begin{proof}

Define $V = \overline O \setminus O $ and let $\epsilon : = 1/2 \inf_{x\in K}{d_Y(x)}$. Since $K$ is compact, it is totally bounded. This is to say that $\exists a_1, a_2, \dots, a_{N_\epsilon}\in K$ such that $K \subset \cup_{i = 1}^{N_\epsilon} B(a_i, \epsilon)$. 

Denote $U = \cup_{i = 1}^{N_\epsilon} B(a_i, \epsilon)$ and $U$ is open. We need to show $\overline U \subset O$. This is true since for any $a_i\in \{a_1, \dots, a_{N_\epsilon}\}$, $\overline B(a_i, \epsilon) \subset O$. With the results from problem 2, $\overline U \subset O$ is proved.

In conclusion, $K\subset U\subset \overline U\subset O$.


\end{proof}



\end{document}