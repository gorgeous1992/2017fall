\documentclass[12pt]{article}
 \usepackage[margin=1in]{geometry} 
\usepackage{amsmath,amsthm,amssymb,amsfonts}
\usepackage{color}
\usepackage{mathtools, eucal}
\usepackage{xparse}
\usepackage{romannum}
 
\DeclarePairedDelimiter\floor{\lfloor}{\rfloor}

 
\begin{document}
 
%\renewcommand{\qedsymbol}{\filledbox}
%Good resources for looking up how to do stuff:
%Binary operators: http://www.access2science.com/latex/Binary.html
%General help: http://en.wikibooks.org/wiki/LaTeX/Mathematics
%Or just google stuff
 
\title{MA 515 Homework 6}
\author{Zheming Gao}
\maketitle

\section*{Problem 1}

\section*{Problem 2}

\begin{proof}

Suppose $x \neq y$. Let $V = span\{x, y\}$ functional $f: V\rightarrow \mathbb R $ such that $\forall s, t\in \mathbb R$, 

$$
f(sx + ty) = s\|x\| - t\|y\|.
$$

Hence, $f(x) = \|x\|$, $f(y) = -\|y\|$ and $f(x)\neq f(y)$. By the theorem, there exists a functional $F: X\rightarrow \mathbb R$ such that $F = f$ on $V$ and $\|f\|_\infty = \|F\|_\infty$, which is a contradiction.

\end{proof}

\section*{Problem 3}


\section*{Problem 7}

\begin{proof}

\begin{enumerate}
\item

$<x,x> = 1/2(\|x+x\|^2 - \|x\|^2 - \|x\|^2) = \|x\|^2 \geqslant 0$. And $<x,x> = 0$ if and only if $x = 0$.

\item 

It is also clear that $<x, y> = <y, x>, \forall x, y \in X$.

\item
We will show $<x, y+z> = <x, y>+<x, z>$, $\forall x, y, z\in X$. By definition, we know

$$
<x, y+z> = \frac{1}{2} (\|x+y+z\|^2 - \|x\|^2 - \|y+z\|^2).
$$
and 
\begin{equation}\label{p7_1}
\begin{aligned}
\|x+y+z\|^2 = & 2\|x\|^2 + 2\|y+z\|^2 - \|x-y-z\|^2 \\
 = & 2\|x+y\|^2 + 2\|z\|^2 - \|x+y-z\|^2
\end{aligned}
\end{equation}

Also, with Parallelogram theorem, we have 

$$
\|x-y-z\|^2 + \|x+y-z\|^2 = 2\|x - z\| + 2\|y\|^2.
$$

Hence, plug it in (\ref{p7_1}) and have 

$$
\|x+y+z\|^2 = \|x\|^2 + \|y+z\|^2 + \|x+y\|^2 + \|z\|^2 - \|x - z\|^2 - \|y\|^2.
$$

which implies

$$
\begin{aligned}
<x, y+z> =& \frac{1}{2} (\|x + y\|^2 - \|x - z\|^2 + \|z\|^2 - \|y\|^2) \\
=& \frac{1}{2} (\|x + y\|^2 - \|y\|^2 - \|x\|^2 - \|x - z\|^2 + \|z\|^2 + \|x\|^2 )\\
 =& <x, y> + <x, z>
\end{aligned}
$$

\item

We need to show that $<\lambda x, y> = \lambda<x, y>$, $\forall x, y\in X, \lambda \in \mathbb R$.

To show this, we need a few steps. Firstly, it holds for $\lambda\in \mathbb N$ and it can be proved by induction. Also, 
$$
\begin{aligned}
<x, -y> =& \frac{1}{2} (\|x - y\|^2  - \|x\|^2 - \|y\|^2) \\
=& \frac{1}{2} (-\|x + y\|^2 + \|y\|^2 + \|x\|^2 )\\
 =& -<x, y> 
\end{aligned}
$$ 

Hence, it holds for $\lambda = -1$ and so holds for $\lambda\in \mathbb Z$. 

Next we will show that it holds for $\lambda \in \mathbb Q$. Let $\lambda = p/q, (q\neq 0), p, q \in \mathbb Z$. Hence, 

$$
q<x, \lambda y> = q <x, \frac{p}{q}y> = <x, py> = p<x, y>.
$$ 

Both sides divided by $q$ and we have $<\lambda x, y> = \lambda<x, y>, \forall \lambda \in \mathbb Q$.

Since $\mathbb Q$ is dense in $\mathbb R$, $\forall \lambda\in\mathbb R$, there exists a sequence of rational numbers $\{\lambda_n\}_{n\in\mathbb N}$ such that $\lambda_n \rightarrow \lambda$. Hence, $<\lambda x, y> = \lambda<x, y>, \forall \lambda \in \mathbb R$.

In conclusion, $<\cdot, \cdot>$ is an inner product.

\end{enumerate}

\end{proof}

\section*{Problem 8}

\begin{enumerate}
\item [(i)]

\begin{proof}

Suppose $\|\cdot\|_1$ is induced by inner product, i.e., $\|\cdot\|_1 = \sqrt{<\cdot, \cdot>}$. However, if so, then $\|\cdot\|$ must satisfy parallelogram identity. For $a = (1,1)^T, b = (-1, 2)^T$, $$
2\|a\|_1^2 + 2\|b\|_1^2 = 26 \neq 18 = \|a - b\|_1^2 + \|a + b\|_1^2.
$$
This is a contradiction.

\end{proof}

\item [(ii)]

Still, it breaks the parallelogram identity.

Let $f(x) = x, g(x) = 2x$. Hence, $\|f\|_\infty = 1, \|g\|_\infty = 2$. But $\|f+g\|_\infty = 3, \|f-g\|_\infty = 1$. So 

$$
\|f+g\|_\infty + \|f-g\|_\infty \neq 2\|f\|_\infty + 2\|g\|_\infty.
$$

\end{enumerate}

\section*{Problem 9}

\begin{proof}

\begin{enumerate}
\item ["$\Rightarrow$"], proved in class. 
 $\left\langle x, x_n \right\rangle$ converges $\left\langle x, x \right\rangle = \|x\|^2$. Hence, 
 
$$
\lim_{n\rightarrow +\infty} \|x_n - x\|^2 = \lim_{n\rightarrow +\infty} \left\langle x_n - x, x_n - x \right\rangle = \lim_{n\rightarrow +\infty} \|x_n\|^2 - 2\left\langle x, x_n \right\rangle + \|x\|^2 = 0
$$

\item ["$\Leftarrow$"]. If $x_n\rightarrow x$, then by Cauchy-Schwartz inequality,

$$
0\leqslant \lim_{n\rightarrow +\infty} \left| \left\langle x_n - x, x \right\rangle\right| \leqslant \lim_{n\rightarrow +\infty} \|x_n - x\| \|x\| = 0.
$$

By squeeze theorem, $\lim_{n\rightarrow +\infty} \left\langle x_n - x, x \right\rangle = 0$. Hence, $x_n \rightharpoonup x$. 

Also, 
$$
\begin{aligned}
0 = \varlimsup_{n\rightarrow +\infty} \|x_n - x\|^2 & = \varlimsup_{n\rightarrow +\infty} \left\langle x_n - x, x_n - x \right\rangle\\
 = &  \varlimsup_{n\rightarrow +\infty} \|x_n\|^2 - 2 \varlimsup_{n\rightarrow +\infty} \left\langle x, x_n \right\rangle + \|x\|^2 \\
= &  \varlimsup_{n\rightarrow +\infty} \|x_n\|^2 - 2 \lim_{n\rightarrow +\infty} \left\langle x, x_n \right\rangle + \|x\|^2 \\
= &  \varlimsup_{n\rightarrow +\infty} \|x_n\|^2 - \|x\|^2
\end{aligned}
$$

Similarly, we have $\varliminf_{n\rightarrow +\infty} \|x_n\|^2 - \|x\|^2 = 0$. Hence, $\lim_{n\rightarrow +\infty} \|x_n\| = \|x\|$.

\end{enumerate}
\end{proof}

\section*{Problem 10}

\begin{proof}

\begin{enumerate}
\item [(i)]

(Shown in class) Since $\{e_n\}_{n\in \mathbb N}$ is an orthonormal basis of $\mathcal H$, for any $x\in \mathcal H$, it can be expressed as

$$
x = \sum_{i=1}^\infty \alpha_i e_i \qquad \alpha_i \in \mathbb R.
$$

Hence, 
$$
\left\langle x, e_i  \right\rangle = \alpha_i \quad \text{and} \quad \left\langle x, x  \right\rangle = \|x\|^2 = \sum_{i=1}^\infty\alpha_i ^2 < +\infty.
$$

Hence, 
$$
\lim_{n\rightarrow +\infty} \left\langle x, e_i  \right\rangle = \lim_{n\rightarrow +\infty} \alpha_i = 0.
$$

i.e., $e_n \rightharpoonup 0$.

\item [(ii)]

Let $\{e_n\}_{n\geqslant1}$ be an orthonormal  basis of $H$ such that $e_n \bot x$, $\forall n\geqslant 2$. Hence, for each $n\in \mathbb N$ and $n\geqslant 2$, let $x_n = x + \lambda e_n (\lambda > 0)$. We want $\|x_n\| = 1$, i.e.,

$$
\|x_n\|^2 = \|x + \lambda e_n\|^2 = \|x\|^2 + \lambda^2 = 1
$$

which yields that $\lambda = \sqrt{1 - \|x\|^2}$. 

Next, we need to show, $x_n = x + \sqrt{1 - \|x\|^2} e_n$ weakly converges to $x$ as $n\rightarrow +\infty$. Indeed, for any arbitrarily taken $y\in H$, use result in (i),

$$
\lim_{n\rightarrow +\infty} \langle y, x_n - x \rangle = \sqrt{1 - \|x\|^2} \langle y, en \rangle = 0
$$

Hence, $x_n \rightharpoonup x$.

\end{enumerate}

\end{proof}



\section*{Problem 11}

\begin{proof}

If $\sum_{n=1}^\infty |\alpha_n|^2 <+\infty $. Let partial sum $S_n = \sum_{i=1}^n \alpha_i v_i$, we need to show that $S_n$ converges as $n\rightarrow +\infty$. Since $H$ is a Hilbert space, it is enough to show $\{S_n\}_{n\geqslant 1}$ is a Cauchy sequence. Indeed, for $m, n \in \mathbb N$, $m > n$, 

$$
\begin{aligned}
\|S_n - S_m\|^2 = \|\sum_{i = n+1}^m \alpha_i v_i\|^2 \leqslant \sum_{i = n+1}^m |\alpha_n|^2.
\end{aligned}
$$

Since $\sum_{n=1}^\infty |\alpha_n|^2 <+\infty $, $\sum_{i = n+1}^m |\alpha_n|^2 \rightarrow 0$ as $m, n\rightarrow +\infty$. Hence, $\{S_n\}$ is Cauchy and so it converges in $H$.

Conversely, if $\{S_n\}$ converges in $H$, and denote the limit as $S$. Hence,  

$$
 +\infty \geqslant \|S\|^2 = \langle \sum_{i = 1}^\infty \alpha_i v_i, \sum_{i = 1}^\infty \alpha_i v_i \rangle = \sum_{n=1}^\infty |\alpha_n|^2.
$$

\end{proof}


\section*{Problem 12}

\begin{enumerate}
\item [(i)]

\begin{proof}

Suppose both $p$ and $q$ are in $\Omega$, such that $p = \pi_\Omega(x), q = \pi_\Omega(x)$. Then, let sequence $\{y_n\}$ be 

$$
y_n = \left\{
\begin{aligned}
p \qquad&  \text{if \ n \ odd}\\
q \qquad&  \text{if \ n \ even}
\end{aligned}\right.
$$

Hence, $\{y_n\}$ converges since $\|y_n - x\|\rightarrow d_\Omega(x)$. And this forces $p = q$.

\end{proof}

\item [(ii)]

\end{enumerate}


\end{document}