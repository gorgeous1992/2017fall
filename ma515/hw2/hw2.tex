\documentclass[12pt]{article}
 \usepackage[margin=1in]{geometry} 
\usepackage{amsmath,amsthm,amssymb,amsfonts}
\usepackage{mathtools, eucal}
\usepackage{xparse}
 
\DeclarePairedDelimiter\floor{\lfloor}{\rfloor}

 
\begin{document}
 
%\renewcommand{\qedsymbol}{\filledbox}
%Good resources for looking up how to do stuff:
%Binary operators: http://www.access2science.com/latex/Binary.html
%General help: http://en.wikibooks.org/wiki/LaTeX/Mathematics
%Or just google stuff
 
\title{MA 515 Homework 2}
\author{Zheming Gao}
\maketitle

\section*{Problem 1}

\begin{enumerate}
\item [(a)]

\begin{proof}

If $\exists \mathcal{O} \in Y$ such that $\mathcal{O}$ is open in $Y$ and $E = \mathcal{O} \cap X$, then $\forall x \in E$, $x$ must be in $\mathcal{O}$. Since $\mathcal{O}$ is open, $\exists r_0$ such that $B_Y(x, r_0) \subset \mathcal{O}$. i.e, $\{y\in Y | d(x, y) < r_0\} \subset \mathcal{O}$. With the fact that $X\subset Y$, $B_X(x, r_0) : = \{ y\in X | d(x, y) < r_0 \} \subset \mathcal{O}$. Hence, $E$ is open in $X$.

Conversely, if $E$ is open in $X$, then for all $x\in E$, there exists $r_x >0$ such that $\{ y\in X | d(x, y) < r_x \} \subset E$. Since $X \subset Y$, let $\mathcal{O}: = \{ y\in Y | d(x, y) < r_x \}$ and it is clear that $\mathcal{O}\cap X \subset E$. What's more, $\forall x \in E$, $x$ is also in $\mathcal{O}\cap X$. Hence, $E = \mathcal{O}\cap X$.

\end{proof}

\item [(b)]

\begin{proof}

If $E$ is closed in X, then $X \setminus E$ is open in $X$. Use the result from (a), it is equivalent to say that there exists an open set $\mathcal{O} \subset Y$ such that $X\setminus E = \mathcal{O} \cap X$. And this implies,

$$
E = X\setminus (O\cap X) = X \setminus \mathcal{O} \cup \phi = X \setminus \mathcal{O} = ( X \setminus \mathcal{O}) \cap Y = (Y\setminus \mathcal{O}) \cap X.
$$

Let $F = Y\setminus \mathcal{O}$ and $F$ is closed in $Y$.

\end{proof}

\end{enumerate}

\section*{Problem 2}

\begin{enumerate}
\item [(a)]

We want to show $\overline{U\cup V} \subset \overline U \cup \overline V$. Take $x \in \overline U$, by definition of closure, $\forall \epsilon >0$, $\exists y \in U\subset (U\cup V)$ such that  $d(x, y) < \epsilon$. This implies that $x \in \overline{U\cup V}$. Hence, $\overline U \subset \overline{U\cup V}$. Similarly, $\overline V \subset \overline{U\cup V}$. In all, $\overline U \cup \overline V \subset \overline{U\cup V}$.

For the other direction, we prove it by contradiction. Suppose $\exists y \in \overline{U \cup V}$ such that $y \notin \overline U \cup \overline V$. Then there exists $\epsilon > 0$, such that, $d(x, y) \geqslant \epsilon$ and $d(x, z) \geqslant \epsilon$ for all $x\in U, z \in V$. i.e, $\forall w in U\cup V$, $d(w, y) > \epsilon$, which is $y\notin \overline{U \cup V}$ (contradiction).

In conclusion, $\overline{U\cup V} = \overline U \cup \overline V$.

\item [(b)]

We want to show $\overline{U \cap V} \subset \overline U \cap \overline V$. If $x\in \overline{U\cap V}$, $\forall \epsilon > 0$, $\exists y \in U\cap V$ such that $d(x, y) < \epsilon$. Since $y y \in U\cap V \Rightarrow y \in U, y \in V$, we know that $x\in \overline U, x \in\overline V$. Hence, $x\in \overline U \cap \overline V$, i.e, $\overline{U \cap V} \subset \overline U \cap \overline V$.

Conversely, we prove it by contradiction. Suppose there exists $x \in \overline U \cap \overline V$, such that $x \notin \overline {U\cap V}$. Then, there exists $\epsilon > 0$ such that $\forall z \in U\cap V$, $d(x, z)\geqslant \epsilon$. Since $z\in U\cap V \Rightarrow z\in \overline U$, it implies that $x\in \overline U$, which leads to a contradiction.

In conclusion, $\overline{U \cap V} = \overline U \cap \overline V$.

\end{enumerate}






\section*{Problem 5}

\begin{proof}

For one direction, if $f$ is not continuous, then by definition, there exists $\epsilon_0 > 0$, for each $n$, $\exists x_n \in X$, such that $|x_n - x| < 1/n$, but $\sigma (f(x_n), f(x)) \geqslant \epsilon_0$. And this implies a contradiction, for then $x_n \rightarrow x$ but $f(x_n)$ doesn't converge to $f(x)$.

For the other direction, we know $f$ is continuous, i.e.,  $\forall \epsilon > 0$, $\exists \delta > 0$, such that $\sigma (f(y) - f(x)) < \epsilon$ holds for all $d(x, y) < \delta$. With $x_n \rightarrow x$, there exists integer $N_\delta > 0$, such that $d(x_n , x) < \delta$ holds for all $n > N_\delta$.

Hence, $\forall \epsilon > 0$, $\exists N_\delta > 0$ such that $\sigma (f(x_n) - f(x)) < \epsilon$ holds for all $n > N_\delta$. And this is equivalent to $f(x_n) \rightarrow f(x)$.

\end{proof}

\section*{Problem 11}

\begin{proof}

\begin{enumerate}
\item Step 1

We want to show that if $E$ is totally bounded, then $\bar E$ is totally bounded. By definition, $\forall \epsilon >0$, there exists finitely many points $a_1, a_2, \dots, a_{N_\epsilon} \in E$ such that $E \subset \cup_{i=1}^{N_\epsilon} B(a_i, \epsilon/2)$. Since $E$ is dense in $\bar E$, $\forall x \in \bar E$, $\exists y \in E$ such that $d(x, y)<\epsilon/2$. This is equivalent that $\exists a \in \{a_1, a_2, \dots, a_{N_\epsilon}\}$ such that $\exists y \in B(a, \epsilon/2)$, and $d(x, y) < \epsilon/2$. With triangle inequality, 

$$
d(a, x) \leqslant d(a, y) + d(y, x) < \epsilon/2 + \epsilon/2 = \epsilon.
$$

Hence, $x$ is in $B(a, \epsilon)$, which leads to $x \in \cup_{i=1}^{N_\epsilon} B(a_i, \epsilon/2)$. Since $x$ is arbitrarily chosen from $\bar E$, we conclude that $\bar E \subset \cup_{i=1}^{N_\epsilon} B(a_i, \epsilon/2)$. i.e, $\bar E$ is also totally bounded.

\item Step 2

Conversely, let $\bar E$ is totally bounded, then $\forall \epsilon > 0$, $\exists a_1, \dots, a_{N_\epsilon}$ such that $\bar E \subset \cup_{i=1}^{N_\epsilon} B(a_i, \epsilon/2)$. With the fact that $E$ is dense in $\bar E$, for any $a_i$, there exists $b_i \in E$ such that $d(b_i, a_i)<\epsilon$. Hence, by triangle inequality, $\forall y \in B(a_i, \epsilon)$, 

$$
d(y, b_i) \leqslant d(y, a_i) + d(a_i, b_i) < \epsilon/2 + \epsilon/2 = \epsilon.
$$

Hence, $y \in B(b_i, \epsilon)$, which implies $\bar E \in \cup_{i=1}^{N_\epsilon} B(b_i, \epsilon)$. Since $E\subset \bar E$ and $b_i \in E, \forall i$, we conclude that $E \subset \cup_{i=1}^{N_\epsilon} B(b_i, \epsilon)$, which is equivalent to that $E$ is totally bounded.


\end{enumerate}

\end{proof}


\end{document}