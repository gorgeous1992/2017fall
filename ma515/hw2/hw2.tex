\documentclass[12pt]{article}
 \usepackage[margin=1in]{geometry} 
\usepackage{amsmath,amsthm,amssymb,amsfonts}
\usepackage{mathtools, eucal}
\usepackage{xparse}
 
\DeclarePairedDelimiter\floor{\lfloor}{\rfloor}

 
\begin{document}
 
%\renewcommand{\qedsymbol}{\filledbox}
%Good resources for looking up how to do stuff:
%Binary operators: http://www.access2science.com/latex/Binary.html
%General help: http://en.wikibooks.org/wiki/LaTeX/Mathematics
%Or just google stuff
 
\title{MA 515 Homework 2}
\author{Zheming Gao}
\maketitle

\section*{Problem 5}

\begin{proof}

For one direction, if $f$ is not continuous, then by definition, there exists $\epsilon_0 > 0$, for each $n$, $\exists x_n \in X$, such that $|x_n - x| < 1/n$, but $\sigma (f(x_n), f(x)) \geqslant \epsilon_0$. And this implies a contradiction, for then $x_n \rightarrow x$ but $f(x_n)$ doesn't converge to $f(x)$.

For the other direction, we know $f$ is continuous, i.e.,  $\forall \epsilon > 0$, $\exists \delta > 0$, such that $\sigma (f(y) - f(x)) < \epsilon$ holds for all $d(x, y) < \delta$. With $x_n \rightarrow x$, there exists integer $N_\delta > 0$, such that $d(x_n , x) < \delta$ holds for all $n > N_\delta$.

Hence, $\forall \epsilon > 0$, $\exists N_\delta > 0$ such that $\sigma (f(x_n) - f(x)) < \epsilon$ holds for all $n > N_\delta$. And this is equivalent to $f(x_n) \rightarrow f(x)$.

\end{proof}


\end{document}