\documentclass[12pt]{article}
 \usepackage[margin=1in]{geometry} 
\usepackage{amsmath,amsthm,amssymb,amsfonts}
\usepackage{color}
\usepackage{mathtools, eucal}
\usepackage{xparse}
\usepackage{romannum}
 
\DeclarePairedDelimiter\floor{\lfloor}{\rfloor}

 
\begin{document}
 
%\renewcommand{\qedsymbol}{\filledbox}
%Good resources for looking up how to do stuff:
%Binary operators: http://www.access2science.com/latex/Binary.html
%General help: http://en.wikibooks.org/wiki/LaTeX/Mathematics
%Or just google stuff
 
\title{MA 515 Homework 5}
\author{Zheming Gao}
\maketitle

\section*{Problem 1}

\begin{proof}

Let $V = \text{span}\{v_1, \dots, v_n\}$, where $v_1, \dots, v_n$ are linearly independent elements in $V$. Then there exists $n$ linearly independent elements $x_1, \dots, x_n \in X$ such that $T(x_i) = v_i, i = 1, \dots, n$. The existence promised by the fact that $T$ is a linear operator. Let $Y_0 = \text{span}\{x_1, \dots, x_n\}$. Hence, $\dim(Y_0) = \dim(V) = n$.

Also, $\ker(T) \cap Y_0 = \{0\}$. Indeed, if $\exists y\neq 0, y\in Y_0$ such that $T(y)=0$. Let $y = \sum_{i=1}^n \beta_i x_i$. Then there exists $\beta_i \neq 0$. By linearity of $T$, $T(y) = \sum_{i=1}^n \beta_i T(x_i) = \sum_{i=1}^n \beta_i v_i \neq 0$ and it yields a contradiction.

Next, we will show $\ker(T) + Y_0 = X$. Suppose not, for any $x\in X$, there exists $z \notin \ker(T) + Y_0, w\in \ker(T), r\in Y_0$ such that $x = z + w + r$. Let $r = \sum_{i=1}^n t_i x_i$, and $T(x) = \sum_{i=1}^n \alpha v_i$.  Hence, 

$$
\sum_{i=1}^n \alpha_i v_i = T(x) = T(z) + T(w) + T(r) = T(z) + \sum_{i=1}^n t_i v_i.
$$, which implies that $T(z) = \sum_{i=1}^n (\alpha_i - t_i) v_i$.

However, $z\notin \ker(T) + Y_0$ and so $T(z)\notin \text{span}\{ v_1, \dots, v_n \}\subset (\ker(T) + Y_0)$. Hence, it is a contradiction.

In conclusion, $\ker(T)+Y_0 = X$ and $\ker(T)\cap Y_0 = \{0\}$, and it implies that $X = \ker(T) \oplus Y_0$.

\end{proof}


\section*{Problem 2}

\begin{proof}

If $T$ is continuous, then the preimage(i.e., $\ker(T)$) of $\{0\}$ is closed since $\{0\}$ is closed. Also, $\ker(T)$ is a subspace due to the linearity of $T$.

If $\ker(T)$ is a closed subspace in $X$, we need to show that $T$ is continuous, or equivalently, bounded. Since $Y$ is a finite-dimensional space, from the result of problem 1, we know there exists a finite-dimensional subspace $Y_0\subset X$ such that $X = \ker(T)\oplus Y_0 $. Hence, for any $x\in X$, there exists $y\in \ker(T), z\in Y_0$ such that $x = y + z$.

Consider the norm of $T$, 

$$
\|T\|_\infty = \sup_{\|x\|\leqslant  1}\|T(x)\| \leqslant\sup_{\|x\|\leqslant  1 \\ y\in \ker} .
$$



...............

................

.................







\end{proof}

\section*{Problem 3}

\begin{proof}

Denote the graph of $f$ as $G(f):= \{ (x, f(x)) | x\in X \} \subset X\times Y$. Let $\{z_n\}_{n\in\mathbb N} = \{ (x_n, f(x_n)) \}_{n\in\mathbb N} \subset G(f)$ that converges to $z = (x, y)$. It is enough to show that $y = f(x)$.

Indeed, $x = \lim_{n\rightarrow +\infty} x_n$ and so $\lim_{n\rightarrow +\infty} f(x_n) = f(x)$ due to the continuity of $f$. Also, $y = \lim_{n\rightarrow +\infty} f(x_n)$. Hence, $y = f(x)$.

\end{proof}

{\color{red}Question:} Here we only need $X, Y$ to be metric spaces. We didn't really need completeness. Is it correct?


\section*{Problem 4}

\begin{enumerate}

\item [(a)]

\begin{proof}

 



\end{proof}

\item [(b)]

Let $f$ be the following function,

$$
f(x) = \left\{
\begin{aligned}
\frac{1}{x}\qquad x \neq 0\\
0 \qquad x = 0.
\end{aligned}
\right.
$$

\end{enumerate}



\section*{Problem 5}

\begin{proof}

$S\circ T$ is a linear operator. Indeed, the domain of $S\circ T$ is $X$, which is a subspace of itself. Also, function composition preserves linearity.

Next we need to show $S\circ T$ is bounded. Consider norm $\|S\circ T\|_\infty$,

$$
\begin{aligned}
\|S\circ T\|_\infty = \sup_{\|x\|_X = 1} \|(S\circ T)(x)\|_Z & = 
\sup_{\|x\|_X = 1} \|S(T(x))\|_Z \\
& \leqslant \sup_{\|x\|_X = 1} \|S\|_\infty \|T(x)\|_Y \\
& = \|S\|_\infty \sup_{\|x\|_X = 1} \|T(x)\|_Y \\
& = \|S\|_\infty \|T\|_\infty.
\end{aligned}
$$

Since both $S$ and $T$ are bounded linear operators, $\|S\|_\infty$ and $\|T\|_\infty$ are less than positive infinity and this leads to the conclusion that $\|S\circ T|_\infty < +\infty$. 

In conclusion, $S\circ T$ is a bounded linear operator.

\end{proof}

\section*{Problem 6}

\begin{proof}

\begin{enumerate}

\item [(a)]

$T$ is a contraction mapping. Indeed, let $ c = \|T\|_\infty < 1$. Hence, for any $x_1, x_2 \in X (x_1 \neq x_2)$, 
$$
\|T(x_1) - T(x_2)\| = \|T(x_1 - x_2)\| \leqslant \|T\|_\infty\|x_1 - x_2\| = c\|x_1 - x_2\|.
$$

where $0<c<1$. Hence, there exists a unique $x_0\in X$ such that $T(x_0) = x_0$. And this is equivalent to say that linear operator (which is easy to check) $ \mathcal N (I-T) = \{x_0\}$. However, $\{0\}\in \mathcal N (I-T)$ always holds. Thus, $x_0 = 0$. 

Next we need to show that $I-T$ is a one-to-one mapping. Suppose not, then there exists $y\in X$ and distinct $y_1, y_2 \in X$ such that $(I-T)(y_1) = (I-T)(y_2) = y $. By linearity, $(I-T)(y_1 - y_2) = 0$ and it yields that $y_1 = y_2$, which is a contradiction. 

What's more, $I-T$ is surjective. Indeed, $I - T$ maps from $X$ to $X$. For any element $x\in X$, $\exists$ unique $y\in X$ such that $y = (I-T)(x)$. Hence, volume of the range of $I-T$ equals to the volume of domain. i.e., $|\mathcal R(I-T)| = |\mathcal D(I-T)| = |X|$. Also, $\mathcal R(I-T) \subset X$, which yields that $\mathcal R(I-T) = X$. In conclusion, $I-T$ is bijective. 

\item [(b)]

Let $S = \sum_{n=0}^\infty T^n$ and consider $\|S\|_\infty$.

$$
\begin{aligned}
\|S\|_\infty & \leqslant \|\lim_{m\rightarrow +\infty}\sum_{n=0}^mT^n\|_\infty \\
& =  \lim_{m\rightarrow +\infty}\|\sum_{n=0}^mT^n\|_\infty \\
& \leqslant \lim_{m\rightarrow +\infty}\sum_{n=0}^m \|T^n\|_\infty \\
& \leqslant \lim_{m\rightarrow +\infty}\sum_{n=0}^m\|T\|^n_\infty \\
& = \lim_{m\rightarrow +\infty}\sum_{n=0}^m c^n \\
& = \frac{1}{1-c} < +\infty.
\end{aligned}
$$

We use triangle inequality above. Also, the limit and norm can exchange due to the continuity of norm. 

Hence, $S$ is bounded in $\|\cdot\|_\infty$. And it is obvious that $S$ is a linear operator, so $S\in (B(X, X), \|\cdot\|_\infty)$.

\item [(c)]

It is enough to check that $S\circ (I-T) = (I-T)\circ S = I$.

$$
\begin{aligned}
S\circ(I-T) = S - S\circ T & = \sum_{n=0}^\infty T^n - \sum_{n=1}^\infty T^n \\
& = T^0 = I
\end{aligned}
$$

$$
\begin{aligned}
(I-T)\circ S =  S- T\circ S  & = \sum_{n=0}^\infty T^n - \sum_{n=1}^\infty T^n \\
& = T^0 = I
\end{aligned}
$$

Hence, $S = (I -T)^{-1}$.

\end{enumerate}
\end{proof}

\section*{Problem 7}

\begin{proof}

For each $n\in\mathbb N$, $\|T^n\|_\infty \leqslant \|T\|^n_\infty$. Since $T$ is a bounded linear operator, $\|T\|_\infty < +\infty$. By triangle inequality and Taylor theorem, 

$$
\|S\|_\infty \leqslant \sum_{n=0}^{+\infty}\frac{\|T^n\|_\infty}{n!} \leqslant \sum_{n=0}^{+\infty}\frac{\|T\|^n_\infty}{n!} = e^{\| T\|_\infty}.
$$

\end{proof}


\end{document}