\documentclass[12pt]{article}
 \usepackage[margin=1in]{geometry} 
\usepackage{amsmath,amsthm,amssymb,amsfonts}
\usepackage{color}
\usepackage{mathtools, eucal}
\usepackage{xparse}
\usepackage{romannum}
 
\DeclarePairedDelimiter\floor{\lfloor}{\rfloor}

 
\begin{document}
 
%\renewcommand{\qedsymbol}{\filledbox}
%Good resources for looking up how to do stuff:
%Binary operators: http://www.access2science.com/latex/Binary.html
%General help: http://en.wikibooks.org/wiki/LaTeX/Mathematics
%Or just google stuff
 
\title{MA 515 Homework 5}
\author{Zheming Gao}
\maketitle

\section*{Problem 1}

\begin{proof}

Let $V = \text{span}\{v_1, \dots, v_n\}$, where $v_1, \dots, v_n$ are linearly independent elements in $V$. Then there exists $n$ linearly independent elements $x_1, \dots, x_n \in X$ such that $T(x_i) = v_i, i = 1, \dots, n$. The existence promised by the fact that $T$ is a linear operator. Let $Y_0 = \text{span}\{x_1, \dots, x_n\}$. Hence, $\dim(Y_0) = \dim(V) = n$.

Also, $\ker(T) \cap Y_0 = \{0\}$. Indeed, if $\exists y\neq 0, y\in Y_0$ such that $T(y)=0$. Let $y = \sum_{i=1}^n \beta_i x_i$. Then there exists $\beta_i \neq 0$. By linearity of $T$, $T(y) = \sum_{i=1}^n \beta_i T(x_i) = \sum_{i=1}^n \beta_i v_i \neq 0$ and it yields a contradiction.

Next, we will show $\ker(T) + Y_0 = X$. Suppose not, for any $x\in X$, there exists $z \notin \ker(T) + Y_0, w\in \ker(T), r\in Y_0$ such that $x = z + w + r$. Let $r = \sum_{i=1}^n t_i x_i$, and $T(x) = \sum_{i=1}^n \alpha v_i$.  Hence, 

$$
\sum_{i=1}^n \alpha_i v_i = T(x) = T(z) + T(w) + T(r) = T(z) + \sum_{i=1}^n t_i v_i.
$$, which implies that $T(z) = \sum_{i=1}^n (\alpha_i - t_i) v_i$.

However, $z\notin \ker(T) + Y_0$ and so $T(z)\notin \text{span}\{ v_1, \dots, v_n \}\subset (\ker(T) + Y_0)$. Hence, it is a contradiction.

In conclusion, $\ker(T)+Y_0 = X$ and $\ker(T)\cap Y_0 = \{0\}$, and it implies that $X = \ker(T) \oplus Y_0$.

\end{proof}


\section*{Problem 2}

\begin{proof}

If $T$ is continuous, then the preimage(i.e., $\ker(T)$) of $\{0\}$ is closed since $\{0\}$ is closed. Also, $\ker(T)$ is a subspace due to the linearity of $T$.

If $\ker(T)$ is a closed subspace in $X$, we need to show that $T$ is continuous, or equivalently, bounded. Since $Y$ is a finite-dimensional space, from the result of problem 1, we know there exists a finite-dimensional subspace $Y_0\subset X$ such that $X = \ker(T)\oplus Y_0 $. Hence, for any $x\in X$, there exists $y\in \ker(T), z\in Y_0$ such that $x = y + z$. 

Also, from $X = \ker(T)\oplus Y_0 $, $\dim (Y_0)$ is finite and $\ker(T)$ is closed, we know that both projection maps $\Pi_{\ker}$ and $\Pi_{Y_0}$ are bounded. $y = \Pi_{\ker}(x)$, $z = \Pi_{Y_0}(x)$.

Consider the norm of $T$. For any $x\in X$, by linearity, $T(x) = T(y) + T(z) = T(z) = T\circ \Pi_{Y_0}(x)$.

$$
\begin{aligned}
\|T\|_\infty = \sup_{x\in X\backslash \{0\}}\frac{\|T(x)\|_Y}{\|x\|_X} & =  \sup_{x\in X\backslash \{0\}}\frac{\|T\circ \Pi_{Y_0}(x)\|_Y}{\|\Pi_{Y_0}(x)\|_X} \frac{\|\Pi_{Y_0}(x)\|_X}{\|x\|_X} 
\end{aligned}
$$

Since $\Pi_{Y_0}$ is bounded, $\frac{\|\Pi_{Y_0}(x)\|_X}{\|x\|_X}<+\infty$ for any $x\in X\backslash\{0\}$. Consider linear operator $T|_{Y_0}: Y_0 \rightarrow Y $ and it is bounded since $\dim(Y_0)<+\infty$. 

$$
\sup_{x\in X\backslash \{0\}}\frac{\|T\circ \Pi_{Y_0}(x)\|_Y}{\|\Pi_{Y_0}(x)\|_X} \leqslant \sup_{y\in Y_0\backslash \{0\}}\frac{\|T|_{Y_0}(y)\|_Y}{\|y\|_X} < +\infty.
$$

Hence, $\|T\|_\infty < +\infty$ and so $T$ is continuous.

\end{proof}

\section*{Problem 3}

\begin{proof}

Denote the graph of $f$ as $G(f):= \{ (x, f(x)) | x\in X \} \subset X\times Y$. Let $\{z_n\}_{n\in\mathbb N} = \{ (x_n, f(x_n)) \}_{n\in\mathbb N} \subset G(f)$ that converges to $z = (x, y)$. It is enough to show that $y = f(x)$.

Indeed, $x = \lim_{n\rightarrow +\infty} x_n$ and so $\lim_{n\rightarrow +\infty} f(x_n) = f(x)$ due to the continuity of $f$. Also, $y = \lim_{n\rightarrow +\infty} f(x_n)$. Hence, $y = f(x)$.

\end{proof}

{\color{red}Question:} Here we only need $X, Y$ to be metric spaces. We didn't really need completeness. Is it correct?


\section*{Problem 4}

\begin{enumerate}

\item [(a)]

\begin{proof}
Prove by contradiction.

Suppose that $f$ is not continuous on $\mathbb R$. Hence, there exists one sequence $\{x_n\}_{n\in\mathbb N}\subset \mathbb R$ that converges to $x$, such that a subsequence $\{x_{n_k}\}_{k\geqslant 1} \subset \{x_n\}_{n\geqslant 1}$, from which $\{f(x_{n_k})\}$ doesn't converge to $f(x)$. 

Since $f$ is bounded, we know that $\{f(x_{n_k})\}$ must have a convergent subsequence, denote as $\{f(x_{n_{k_l}})\}_{l\geqslant 1} \rightarrow y$. Also, we know $\{x_{n_{k_l}}\} \rightarrow x$ and with closeness of $G(f)$, we know $y = f(x)$. This is equivalent to say that  $\{f(x_{n_{k_l}})\}_{l\geqslant 1} \rightarrow f(x)$. It leads to a contradiction to the assumption that $\{f(x_{n_k})\}$ doesn't converge to $f(x)$.

In conclusion, $f$ is a continuous function.

\end{proof}

\item [(b)]

Let $f$ be the following function,

$$
f(x) = \left\{
\begin{aligned}
\frac{1}{x}\qquad x \neq 0\\
0 \qquad x = 0.
\end{aligned}
\right.
$$

\end{enumerate}



\section*{Problem 5}

\begin{proof}

\begin{enumerate}
\item [$\Rightarrow$]

If $T_1$ is compact and $T_2$ is continuous, then, for any bounded sequence $\{x_n\}_{n\in \mathbb N} \subset X$, there exists a subsequence $\{x_{n_k}\} \subset \{x_n\}$ such that $\{y_{n_k}\} := \{T_1(x_{n_k})\}$ converges in $Y$. Denote $\lim_{k\rightarrow +\infty} y_{n_k} = y$. Since $T_2$ is continuous, $T_2(y_{n_k}) \rightarrow T_2(y)\in Z$. i.e., for any bounded sequence $\{x_n\}_{n\in \mathbb N} \subset X$, there exists a subsequence $\{x_{n_k}\} \subset \{x_n\}$ such that $\{T_2\circ T_1(x_{n_k})\}$ converges in $Z$. In conclusion, $T_2\circ T_1$ is compact.

\item [$\Leftarrow$]

Conversely, suppose $T_2$ is compact and $T_1$ is continuous. Then, for any bounded set $U\in X$, $T_1(U) = V$ is also a bounded set in $Y$ since $T_1$ is bounded. By compactness, $\overline{T_2(V)}$ is compact. Hence, for any bounded set $U\in X$, $T_2\circ T_1(U)$ is compact in $Z$. Hence, $T_2\circ T_1$ is compact.

\end{enumerate}

\end{proof}


\section*{Problem 6}

\begin{enumerate}

\item [(i)]

\begin{proof}

T is linear, bounded and bijective so that $T^{-1}$ exists and bounded. Note that $\|x\| = \|T^{-1}\circ T(x)\|$. Hence, there exists $M>0$, such that

$$
\frac{\| x \|}{\| T(x) \|} = \frac{\|T^{-1}\circ T(x)\|}{\| T(x) \|} \leqslant \sup_{y\in X\backslash \{ 0 \} } \frac{\| T^{-1} (y) \|}{\| y \|} \leqslant M.
$$

Let $\beta = 1/M$, then $\|T(x)\| \geqslant \beta \|x\|, \forall x \in X$.

\end{proof}

\item [(ii)]

\begin{proof}

From (i) we know that $T^{-1}$ exists as a linear bounded operator. $\|T^{-1}\| \leqslant M = 1/\beta$. 

For each $y\in Y$, we define a map $Q_y: X\rightarrow X$ such that $Q_y(x) : = T^{-1}(y - \Psi(x))$, where $\Psi$ is a bounded linear operator with norm $\|\Psi\|_\infty < \beta$. It is enough to show that $Q_y$ is a contraction mapping.

Indeed, for any $x_1, x_2\in X$, we have

$$
\begin{aligned}
\| Q_y(x_1) - Q_y(x_2) \| = \|T^{-1}(y - \Psi(x_1)) - T^{-1}(y - \Psi(x_2))\| & \leqslant \|T^{-1}\|_\infty \|y - \Psi(x_1) - (y - \Psi(x_2))\| \\
& = \|T^{-1}\| \|\Psi(x_1) - \Psi(x_2)\| \\
& \leqslant \|T^{-1}\|_\infty \|\Psi\|_\infty \|x_1 - x_2\| \\
& = c \|x_1 - x_2\|.
\end{aligned}
$$

where $c = \|T^{-1}\|_\infty \|\Psi\|_\infty \in (0, 1)$. This yields that $Q_y$ is a contraction mapping for each $y \in Y$.

Hence, equation $x = Q_y(x)$ has a unique solution for each $y\in Y$.

\end{proof}


\end{enumerate}

\section*{Problem 7}

\begin{enumerate}

\item [(i)]

\begin{proof}

For given $y\in Y$, define $f_y: X \rightarrow G(B)\cap (X\times\{y\}) \subset (X\times Y)$, $f_y(x) = B(x, y)$. Hence, $f_y$ is linear since $B$ is bilinear. Also, we know that $X\times Y$ is a Banach space since both $X$ and $Y$ are Banach. By closed graph theorem, it is enough to show that $G(f_y)$ is closed.

Take a convergent sequence $\{a_n\}:= \{(x_n, f_y(x_n))\} \subset G(f_y)$ and $a_n \rightarrow a = (a_1, a_2)$. i.e., $\lim_{n\rightarrow +\infty} x_n = a_1, \lim_{n\rightarrow +\infty} f_y(x_n) = a_2$. We want to show that $f_y(a_1) = a_2$. Indeed, since $B$ is continuous at the origin, with bilinearity, $B$ is continuous on $X\times Y$. Let $y_n = y, \forall n\in \mathbb N$, then the sequence $\{(x_n, y_n)\}\rightarrow (a_1, y)$. Hence, $B(x_n, y_n) = B(x_n, y) = f_y(x_n)$ converges to $B(a_1, y)$, which is $f_y(a_1)$. Hence, $a_2 = f_y(a_1)$, and so $G(f_y)$ is closed.

In conclusion, $f_y$ is bounded. 

\end{proof}


\item[(ii)]



\end{enumerate}


\end{document}