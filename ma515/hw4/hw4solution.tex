\documentclass[12pt]{article}
 \usepackage[margin=1in]{geometry} 
\usepackage{amsmath,amsthm,amssymb,amsfonts}
\usepackage{color}
\usepackage{mathtools, eucal}
\usepackage{xparse}
\usepackage{romannum}
 
\DeclarePairedDelimiter\floor{\lfloor}{\rfloor}

 
\begin{document}
 
%\renewcommand{\qedsymbol}{\filledbox}
%Good resources for looking up how to do stuff:
%Binary operators: http://www.access2science.com/latex/Binary.html
%General help: http://en.wikibooks.org/wiki/LaTeX/Mathematics
%Or just google stuff
 
\title{MA 515 Homework 4}
\author{Zheming Gao}
\maketitle

\section*{Problem 1}

\begin{proof}

$\{T_n\}_{n\in\mathbb N}$ is a sequence of uniformly bounded linear operators and it satisfies 
$$
\lim_{n\rightarrow +\infty} T_n(x) : = T(x)
$$ 
for any $x\in X$. Now we want to show $T$ is a bounded linear operator. First, $T$ is linear because the limit operation on $T_n$ preserves the linearity of $T_n$. Also, $\mathcal{D}(T)$ is  $X$, which yields that $T$ is a linear operator from $X$ to $Y$.

Next we need to show $T$ is bounded, or more precisely, $||T||_\infty \leqslant M$. Since $||T_n||_\infty < M$, we know for any $||x||_X = 1$, $||T_n(x)||_Y < M$. Hence, 

$$
\lim_{n\rightarrow +\infty} \left \|T_n(x)\right\|  = ||T(x)|| \leqslant M.
$$

which implies $\sup_{||x||_X = 1} ||T(x)|| \leqslant M$, i.e., $||T||_\infty \leqslant M$.

\end{proof}


\section*{Problem 2}

\begin{proof}

First we need to show that $\Lambda$ is bounded. Take arbitrarily $x = \{ x_n \}_{n\in\mathbb N}\in\ell^\infty$, and there exists $M>0$ such that $|x_i| \leqslant M$ for any $i \geqslant 0$. Therefore, 

$$
\begin{aligned}
||\Lambda(x)||_{\ell^\infty} = ||y||_{\ell^\infty} & = \sup_{i\geqslant 1} |y_i| \\
& = \sup_{i\geqslant 1} \left|\frac{x_1 + \cdots + x_i}{i}\right| \\
& \leqslant \sup_{i\geqslant 1} \left|\frac{\sum_{j=1}^i|x_j|}{i}\right| \leqslant M.
\end{aligned}
$$ 

Hence, $||\Lambda||_\infty = \sup_{||x||_{\ell^\infty} = 1} ||\Lambda(x)||_{\ell^\infty} \leqslant M$.

Next we need to find the value of $||\Lambda||_\infty$. From the definition, 
$$
||\Lambda||_\infty = \sup_{||x||_{\ell^\infty} = 1} ||\Lambda(x)||_{\ell^\infty} = \sup_{||x||_{\ell^\infty} = 1}\sup_{i\geqslant 1}\left| \frac{x_1 + \cdots + x_i}{i} \right|.
$$

Also, $||x||_{\ell^\infty} = 1$ implies $|x_j| \leqslant 1, \forall j\geqslant 1$. Hence, 

$$
\sup_{||x||_{\ell^\infty} = 1}\sup_{i\geqslant 1}\left| \frac{x_1 + \cdots + x_i}{i} \right| = \sup_{i\geqslant 1} \frac{i\cdot 1}{i} = 1.
$$

Hence, $||\Lambda||_\infty = 1$.

\end{proof}

\section*{Problem 3}

\begin{proof}

It is enough to show that $\mathcal N(T) = \{0\}$. If so, then $T$ is invertible and is a linear operator. Indeed, supposed there exists $x\neq 0, x\in \mathcal N(T)$, then $0 = \|T(x)\| \geqslant \|x\| >0$, which yields a contradiction.

Next we need to show $T^{-1}$ is bounded. Since $T$ is surjective and invertible, we know $T$ must be a bijection. If so, for any $y\in Y$, we have $\|y\|\geqslant b\|T^{-1}(y)\|$. Hence, 

$$
\sup_{\|y\|\neq 0} \frac{\|T^{-1}(y)\|}{\|y\|} \leqslant \frac{1}{b} < +\infty.
$$

i.e., $T^{-1}$ is a bounded linear operator.

\end{proof}


\section*{Problem 4}

\begin{proof}

We want to show $\| T_n(x_n) - T(x) \|_Y \rightarrow 0$ as $n \rightarrow +\infty$. By triangle inequality,


$$
\begin{aligned}
\|T_n(x_n) - T(x)\|_Y & \leqslant \|T_n(x_n) - T(x_n)\|_Y + \|T(x_n) - T(x)\|_Y \\ 
& \leqslant \|T_n- T\|_\infty \|(x_n)\|_X + \|T\|_\infty \|x_n - x\|_X 
\end{aligned}
$$

and we know $\|x_n - x\| \rightarrow 0$ as $n\rightarrow +\infty$. Hence, $\|x_n\|_X$ is bounded. What's more, $\lim_{n\rightarrow +\infty} \|T_n - T\|_\infty = 0$.

In conclusion, $\lim_{n\rightarrow +\infty} ||T_n(x_n) - T(x)|| = 0$.

\end{proof}

\section*{Problem 5}

\begin{proof}

$S\circ T$ is a linear operator. Indeed, the domain of $S\circ T$ is $X$, which is a subspace of itself. Also, function composition preserves linearity.

Next we need to show $S\circ T$ is bounded. Consider norm $\|S\circ T\|_\infty$,

$$
\begin{aligned}
\|S\circ T\|_\infty = \sup_{\|x\|_X = 1} \|(S\circ T)(x)\|_Z & = 
\sup_{\|x\|_X = 1} \|S(T(x))\|_Z \\
& \leqslant \sup_{\|x\|_X = 1} \|S\|_\infty \|T(x)\|_Y \\
& = \|S\|_\infty \sup_{\|x\|_X = 1} \|T(x)\|_Y \\
& = \|S\|_\infty \|T\|_\infty.
\end{aligned}
$$

Since both $S$ and $T$ are bounded linear operators, $\|S\|_\infty$ and $\|T\|_\infty$ are less than positive infinity and this leads to the conclusion that $\|S\circ T|_\infty < +\infty$. 

In conclusion, $S\circ T$ is a bounded linear operator.

\end{proof}

\section*{Problem 6}

\begin{proof}

\begin{enumerate}

\item [(a)]

$T$ is a contraction mapping. Indeed, let $ c = \|T\|_\infty < 1$. Hence, for any $x_1, x_2 \in X (x_1 \neq x_2)$, 
$$
\|T(x_1) - T(x_2)\| = \|T(x_1 - x_2)\| \leqslant \|T\|_\infty\|x_1 - x_2\| = c\|x_1 - x_2\|.
$$

where $0<c<1$. Hence, there exists a unique $x_0\in X$ such that $T(x_0) = x_0$. And this is equivalent to say that linear operator (which is easy to check) $ \mathcal N (I-T) = \{x_0\}$. However, $\{0\}\in \mathcal N (I-T)$ always holds. Thus, $x_0 = 0$. 

Next we need to show that $I-T$ is a one-to-one mapping. Suppose not, then there exists $y\in X$ and distinct $y_1, y_2 \in X$ such that $(I-T)(y_1) = (I-T)(y_2) = y $. By linearity, $(I-T)(y_1 - y_2) = 0$ and it yields that $y_1 = y_2$, which is a contradiction. 

What's more, $I-T$ is surjective. Indeed, $I - T$ maps from $X$ to $X$. For any element $x\in X$, $\exists$ unique $y\in X$ such that $y = (I-T)(x)$. Hence, volume of the range of $I-T$ equals to the volume of domain. i.e., $|\mathcal R(I-T)| = |\mathcal D(I-T)| = |X|$. Also, $\mathcal R(I-T) \subset X$, which yields that $\mathcal R(I-T) = X$. In conclusion, $I-T$ is bijective. 

\item [(b)]

Let $S = \sum_{n=0}^\infty T^n$ and consider $\sum_{n=0}^\infty\|T^n\|_\infty$. Since $X$ is a Banach space, by the theorem, $B(X, X)$ is also a Banach space. Hence, absolutely convergence implies convergence. It is enough to show that $\sum_{n=0}^\infty\|T^n\|_\infty$ exists.

$$
\begin{aligned}
\lim_{m\rightarrow +\infty}\sum_{n=0}^m \|T^n\|_\infty & \leqslant \lim_{m\rightarrow +\infty}\sum_{n=0}^m\|T\|^n_\infty \\
& = \lim_{m\rightarrow +\infty}\sum_{n=0}^m c^n \\
& = \frac{1}{1-c} < +\infty.
\end{aligned}
$$

We use triangle inequality above. Also, the limit and norm can exchange due to the continuity of norm. Hence, $\sum_{n=0}^\infty\|T^n\|_\infty$ exists.

Hence, $S = \lim_{m\rightarrow +\infty} \sum_{n=0}^m T^n$ exists and so it is bounded because $\|S\|_\infty = \|\sum_{n=0}^\infty T^n \|_\infty \leqslant 1/(1-c)$. S is also linear.

\item [(c)]

It is enough to check that $S\circ (I-T) = (I-T)\circ S = I$. Since $S$ exists, we have $\lim_{N\rightarrow +\infty} T^{N} = 0$.

$$
\begin{aligned}
S\circ(I-T) = S - S\circ T & = \lim_{N\rightarrow +\infty}\sum_{n=0}^N T^n - \sum_{n=1}^{N+1} T^n \\
& = \lim_{N\rightarrow +\infty} I - T^{N+1} = I.
\end{aligned}
$$


$(I - T)\circ S$ is the same. Hence, $S = (I - T)^{-1}$.

\end{enumerate}
\end{proof}

\section*{Problem 7}

\begin{proof}

For each $n\in\mathbb N$, $\|T^n\|_\infty \leqslant \|T\|^n_\infty$. Since $T$ is a bounded linear operator, $\|T\|_\infty < +\infty$. By triangle inequality and Taylor theorem, 

$$
\|S\|_\infty \leqslant \sum_{n=0}^{+\infty}\frac{\|T^n\|_\infty}{n!} \leqslant \sum_{n=0}^{+\infty}\frac{\|T\|^n_\infty}{n!} = e^{\| T\|_\infty}.
$$

\end{proof}


\end{document}