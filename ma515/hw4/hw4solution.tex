\documentclass[12pt]{article}
 \usepackage[margin=1in]{geometry} 
\usepackage{amsmath,amsthm,amssymb,amsfonts}
\usepackage{color}
\usepackage{mathtools, eucal}
\usepackage{xparse}
\usepackage{romannum}
 
\DeclarePairedDelimiter\floor{\lfloor}{\rfloor}

 
\begin{document}
 
%\renewcommand{\qedsymbol}{\filledbox}
%Good resources for looking up how to do stuff:
%Binary operators: http://www.access2science.com/latex/Binary.html
%General help: http://en.wikibooks.org/wiki/LaTeX/Mathematics
%Or just google stuff
 
\title{MA 515 Homework 4}
\author{Zheming Gao}
\maketitle

\section*{Problem 1}

\begin{proof}

$\{T_n\}_{n\in\mathbb N}$ is a sequence of uniformly bounded linear operators and it satisfies 
$$
\lim_{n\rightarrow +\infty} T_n(x) : = T(x)
$$ 
for any $x\in X$. Now we want to show $T$ is a bounded linear operator. First, $T$ is linear because the limit operation on $T_n$ preserves the linearity of $T_n$. Also, $\mathcal{D}(T)$ is  $X$, which yields that $T$ is a linear operator from $X$ to $Y$.

Next we need to show $T$ is bounded, or more precisely, $||T||_\infty \leqslant M$. Since $||T_n||_\infty < M$, we know for any $||x||_X = 1$, $||T_n(x)||_Y < M$. Hence, 

$$
\lim_{n\rightarrow +\infty} \left \|T_n(x)\right\|  = ||T(x)|| \leqslant M.
$$

which implies $\sup_{||x||_X = 1} ||T(x)|| \leqslant M$, i.e., $||T||_\infty \leqslant M$.

\end{proof}


\section*{Problem 2}

\begin{proof}

First we need to show that $\Lambda$ is bounded. Take arbitrarily $x = \{ x_n \}_{n\in\mathbb N}\in\ell^\infty$, and there exists $M>0$ such that $|x_i| \leqslant M$ for any $i \geqslant 0$. Therefore, 

$$
\begin{aligned}
||\Lambda(x)||_{\ell^\infty} = ||y||_{\ell^\infty} & = \sup_{i\geqslant 1} |y_i| \\
& = \sup_{i\geqslant 1} \left|\frac{x_1 + \cdots + x_i}{i}\right| \\
& \leqslant \sup_{i\geqslant 1} \left|\frac{\sum_{j=1}^i|x_j|}{i}\right| \leqslant M.
\end{aligned}
$$ 

Hence, $||\Lambda||_\infty = \sup_{||x||_{\ell^\infty} = 1} ||\Lambda(x)||_{\ell^\infty} \leqslant M$.

Next we need to find the value of $||\Lambda||_\infty$. From the definition, 
$$
||\Lambda||_\infty = \sup_{||x||_{\ell^\infty} = 1} ||\Lambda(x)||_{\ell^\infty} = \sup_{||x||_{\ell^\infty} = 1}\sup_{i\geqslant 1}\left| \frac{x_1 + \cdots + x_i}{i} \right|.
$$

Also, $||x||_{\ell^\infty} = 1$ implies $|x_j| \leqslant 1, \forall j\geqslant 1$. Hence, 

$$
\sup_{||x||_{\ell^\infty} = 1}\sup_{i\geqslant 1}\left| \frac{x_1 + \cdots + x_i}{i} \right| = \sup_{i\geqslant 1} \frac{i\cdot 1}{i} = 1.
$$

Hence, $||\Lambda||_\infty = 1$.

\end{proof}

\section*{Problem 3}




\section*{Problem 4}

\begin{proof}

We want to show $\| T_n(x_n) - T(x) \|_Y \rightarrow 0$ as $n \rightarrow +\infty$. By triangle inequality,
$$
\|T_n(x_n) - T(x)\|_Y \leqslant \|T_n(x_n) - T(x_n)\|_Y + \|T(x_n) - T(x)\|_Y \leqslant \|T_n(x_n) - T(x_n)\|_Y + \|T\|_\infty \|x_n - x\|_X.
$$

and we know $\|x_n - x\| \rightarrow 0$ as $n\rightarrow +\infty$. Hence, it is enough to show that $\|T_n(x_n) - T(x_n)\|_Y\rightarrow 0$ as $n\rightarrow +\infty$. 

$\forall x_m\in X, m = 1, 2, \cdots$, $\lim_{n\rightarrow +\infty} ||T_n(x_m) - T(x_m)|| = 0$ since $\{T_n\}$ converges to $T$. Hence, we have 

$$
\lim_{m\rightarrow +\infty} \lim_{n\rightarrow +\infty} ||T_n(x_m) - T(x_m)|| = 0.
$$

This implies $\|T_n(x_n) - T(x_n)\|_Y\rightarrow 0$ because $\{\|T_n(x_n) - T(x_n)\|\}_{n\in\mathbb{N}}$ is a subsequence of $\{\|T_n(x_m) - T(x_m)\|\}_{m, n\in\mathbb{N}}$.

In conclusion, $\lim_{n\rightarrow +\infty} ||T_n(x_n) - T(x_n)|| = 0$.

\end{proof}


\end{document}